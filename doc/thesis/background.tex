\section{RDMA}

  \todo{References: $https://www.mellanox.com/related-docs/prod_software/RDMA_Aware_Programming_user_manual.pdf$}

Remote Direct Memory Access (RDMA) is a network feature that enables direct access to the memory of a remote computer, 
without any interaction by the remote CPU. The complete bypassing of the hosts kernel and even CPU allows us to achieve 
very low latency and high bandwidth, while reducing or eliminating CPU utilization.

While initially developed as part of the \emph{InfiniBand}~\cite{} network protocol used primarily in high performance computing,
it is also available for commodity ethernet using \emph{Internet Wide Area RDMA (iWARP)}~\cite{} or
\emph{RDMA over Converged Ethernet (RoCE)}~\cite{}. For the rest of the thesis we will focus on RoCE which is designed for 
intra data center communication and seems to be adopted the most in modern data centers. \comment{such claims need references}
While we focus on and will only evaluate RoCE, most of this work should be applicable to the other technologies.



\subsection{Verbs API}
\begin{figure}[!ht]
\begin{center}
\begin{tikzpicture}[node distance=2cm,auto,>=stealth']
  \queue[Send Queue]{0,3};
  \queue[Receive Queue]{0,1.5};
  \rqueue[Completion Queue]{0,0};

  \draw[rounded corners] (-.5, -1.5) rectangle (5.5, 3.5) {};
  \node[align=center] at (3cm, 3.2cm) {RNIC};
  \draw[rounded corners] (3.4, -.5) rectangle (5.1, 2.5) {};
  \node[align=center] at (4.25cm, 1cm) {Processing \\ Unit};
\end{tikzpicture}
\end{center}
\caption{Resources of the Verbs API}
\label{fig:rdma-parts}
\end{figure}


All three technologies RoCE, iWARP, and InfiniBand share a common user API called \emph{Verbs API}. The verbs API gives us a
userspace library called \emph{libibverbs} which gives developers direct low level access to the device, bypassing the kernel.

\paragraph{}The verbs API is different from traditional socket programming. Applications interact directly with the Network 
Interface Card (NIC) through so called \emph{Queue Pairs (QP)} and \emph{Completion Queues (CQ)}

A QP consists of two queues that are responsible to schedule work for the NIC. The \emph{Send Queue} and the \emph{Receive 
Queue}. The Completion Queue is used to notify the application when a \emph{Work Request} placed on one of the queues is 
completed.

All

\todo{Give intro to RDMA programming model / ibverbs}

\todo{Use https://zcopy.wordpress.com/tag/rdma/ as inspiration}

\todo{Explain QP, WRs, CQs. Stick to RC.}

\subsubsection{Send / Receive}
The \emph{Send} and \emph{Receive} verbs are the most traditional operations.



\begin{figure}[!ht]
\begin{center}
\begin{tikzpicture}[node distance=2cm,auto,>=stealth']
  \seqnode{B_cpu}{RAM};
  \seqnode[left of=B_cpu]{B_nic}{NIC};
  \hseqnode[right of=B_cpu, node distance=1.5cm]{B_acpu}{};
  \seqnode[left of=B_cpu, node distance=7cm]{A_cpu}{CPU / RAM};
  \seqnode[right of=A_cpu]{A_nic}{NIC};
  \msg{A_cpu}{A_nic}{.25}{WR MMIO}
  \msg[below]{A_cpu}{A_nic}{.3}{payload DMA}
  \msg{A_nic}{B_nic}{.5}{network transfer}
  \msg{B_nic}{B_cpu}{.65}{payload DMA}
  \msg[below]{B_nic}{B_cpu}{.7}{CQE DMA}
  \fetch{B_acpu}{B_cpu}{.8}{poll CQ}
\end{tikzpicture}
\end{center}
\caption{Send Receive sequence}
\label{fig:seq-sndrcv}
\end{figure}





\subsubsection{Write}
\subsubsection{Read}
\subsubsection{Atomics}
