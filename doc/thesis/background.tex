\section{RDMA}

Remote Direct Memory Access (RDMA) is a network feature that enables direct access to the memory of a remote computer, 
without any interaction by the remote CPU. The complete bypassing of the hosts kernel and even CPU allows us to achieve 
very low latency and high bandwidth, while reducing or eliminating CPU utilization.

While initially developed as part of the \emph{InfiniBand}~\cite{} network protocol used primarily in high performance computing,
it is also available for commodity ethernet using \emph{Internet Wide Area RDMA (iWARP)}~\cite{} or
\emph{RDMA over Converged Ethernet (RoCE)}~\cite{}. For the rest of the thesis we will focus on RoCE which is designed for 
intra data center communication and seems to be adopted the most in modern data centers. \comment{such claims need references}
While we focus on and will only evaluate RoCE, most of this work should be applicable to the other technologies.



\subsection{Verbs API}\label{sec:bg:verbs}

All three technologies RoCE, iWARP, and InfiniBand share a common user API called \emph{Verbs API}. The verbs API gives us a
userspace library called \emph{libibverbs} which gives developers direct low level access to the device, bypassing the kernel.

\paragraph{}The verbs API is different from traditional socket programming. Applications interact directly with the Network 
Interface Card (NIC) through so called \emph{Queue Pairs (QP)} and \emph{Completion Queues (CQ)}, allowing it to issue
so called \emph{Verbs}, different operations that the NIC can perform. The verbs are:

\begin{itemize}
  \item Send (with Immediate): Sends a data from the sender memory to a prepared memory region at the receiver
  \item Receive: Prepares a memory region to receive data through the send verb
  \item Write (with Immediate): Copies data from the sender memory to known memory location at the receiver without any 
    interaction from the remote CPU.
  \item Read: Copies data from remote memory to a local buffer without any inteaction from the remote CPU.
  \item Atomics: Two different atomic operations. Copare and Swap (CAS) and Fetch and Add (FAA). They can access 64-bit 
    values in the remote memory. 
\end{itemize}



\begin{figure}[!ht]
\begin{center}
\begin{tikzpicture}[node distance=2cm,auto,>=stealth']
  \queue[Send Queue]{0,3};
  \queue[Receive Queue]{0,1.5};
  \rqueue[Completion Queue]{0,0};

  \draw[rounded corners] (-.5, -1.5) rectangle (5.5, 3.5) {};
  \node[align=center] at (3cm, 3.2cm) {RNIC};
  \draw[rounded corners] (3.4, -.5) rectangle (5.1, 2.5) {};
  \node[align=center] at (4.25cm, 1cm) {Processing \\ Unit};
\end{tikzpicture}
\end{center}
\caption{Resources of the Verbs API}
\label{fig:rdma-parts}
\end{figure}


\paragraph{} A QP consists of two queues that are responsible to schedule work for the NIC. The \emph{Send Queue} and the \emph{Receive 
Queue}. A Work Request (WR) consists of a \code{opcode} which signifies which verb we want to execute, all necessary 
information to complete this operations, and contains a user provided Work Request ID \code{wr\_id}. As soon as the NIC has 
processed and complete the issued work Request it enqueues a Completion Queue Event (CQE) into an other queue called the
Completion Queue (CQ). The CQE will contain the user provided \code{wr\_id} and allows us to notice when a request was 
completed.\comment{Say something on in order guarantee?}


\paragraph{} It is worth noting that every buffer that is accessed by the NIC needs to be previously registered as a usable
\emph{Memory Region (MR)}.

\paragraph{} \todo{Maybe present basic API? ibv\_post\_send, ibv\_post\_receive, ibv\_poll\_cq?}

\todo{Talk about unsignaled WRs}


\subsubsection{Send / Receive} \label{sec:bg:send}
The \emph{Send} and \emph{Receive} verbs are the most traditional operations, which allows us to send a single message to 
a receiver. Let's walk through sending a message.

\begin{figure}[!ht]
\begin{center}
\begin{tikzpicture}[node distance=2cm,auto,>=stealth']
  \node[align=center] at (-6.4,1) {System A};
  \draw[rounded corners] (-9, -6) rectangle (-3.8, 1.5) {};
  \node[align=center] at (-0.6,1) {System B};
  \draw[rounded corners] (-3.2, -6) rectangle (2, 1.5) {};
  \seqnode{B_cpu}{RAM};
  \seqnode[left of=B_cpu]{B_nic}{NIC};
  \hseqnode[right of=B_cpu, node distance=1.5cm]{B_acpu}{};
  \seqnode[left of=B_cpu, node distance=7cm]{A_cpu}{CPU / RAM};
  \seqnode[right of=A_cpu]{A_nic}{NIC};
  \node[align=center, circle, draw=black, minimum size=.5mm] at (-1,-0.4) {\small 1};
  \msg{B_cpu}{B_nic}{.2}{WR MMIO}
  \node[align=center, circle, draw=black, minimum size=.5mm] at (-5.7,-0.7) {\small 2};
  \msg{A_cpu}{A_nic}{.25}{WR MMIO}
  \msg[below]{A_cpu}{A_nic}{.3}{payload DMA}
  \msg{A_nic}{B_nic}{.5}{network transfer}
  \node[align=center, circle, draw=black, minimum size=.5mm] at (-4.5,-2) {\small 3};
  \msg{B_nic}{B_cpu}{.65}{payload DMA}
  \msg[below]{B_nic}{B_cpu}{.7}{CQE DMA}
  \node[align=center, circle, draw=black, minimum size=.5mm] at (-1.3,-2.6) {\small 4};
  \msg[below]{B_nic}{A_nic}{.71}{Acknowledgement}
  \node[align=center, circle, draw=black, minimum size=.5mm] at (-4.5,-4.3) {\small 5};
  \msg[below]{A_nic}{A_cpu}{.76}{DMA CQE}
  \node[align=center, circle, draw=black, minimum size=.5mm] at (0.5,-4.5) {\small 6};
  \fetch{B_acpu}{B_cpu}{.8}{poll CQ}
\end{tikzpicture}
\end{center}
\caption{Send Receive sequence}
\label{fig:seq-sndrcv}
\end{figure}


Let's assume that that system A and B have set up a connection. Each of them have setup a QP and associated a Completion 
Queue to it. Both systems have registered a MR of at least the size of the to be sent message.

\begin{enumerate}
  \item First system B will have to post a \emph{Receive Buffer}, meaning it has to reserve a buffer for incoming messages.
    It does this by moving a Work Request into its Receive Queue. This WR will contain a pointer to the MR he prepared. We
    call this posting a receive buffer. System B will now poll its CQ until it receives a CQE for its issued receive request.
  \item Now system A will initiate the transfer by posting a \emph{Send Request}. It copies a Work Request to the Send 
    Queue, which contains the \code{IBV\_WR\_SEND} opcode and a pointer to its local buffer containing the to be send message
    and its size. It will then also start polling its CQ to notice the completion of the send request.
  \item The previous step was actually an MMIO operation, writing the Work Request to the NIC, which will now 
    accesses the messages payload using DMA. This  might generate more than one PCIe transaction \cite{atc16-kalia}. 
    The payload is then sent over the network.
  \item As soon as the receiver receives the first segment it will consume the posted receive buffer, write incoming payload 
    to the receive buffer and generate a \emph{Work Completion Event (WQE)}. 
  \item The successful writing of the message will generate an acknowledgement, which will be sent back to the sender where 
    its NIC will generate a CQE for the send request, which the sending CPU will be able to poll.
  \item At last the receiving CPU will be able to poll its \emph{Completion Queue (CQ)} and the transfer is complete.
\end{enumerate}

So when we get over the rather unusual interface through these three queues and the rather tedious manual management of 
memory regions, the send and receive verbs are quite simple. They give us a fairly clean message passing notion.


\subsubsection{Write} \label{sec:bg:write}

The \emph{RDMA write} verb shows us the potential of RDMA operations. It allows us to write at an arbitrary, but previously 
by the remote machine registered, memory region, without any interaction of the remote CPU. That also means the remote CPU
is not notified of the operation and no CQE is generated at the remote. This is the main reason why RDMA writes are generally
considered to be faster then send and receive~\cite{anuj-guide}.


\begin{figure}[!ht]
\begin{center}
\begin{tikzpicture}[node distance=2cm,auto,>=stealth']
  \node[align=center] at (-6.4,1) {System A};
  \draw[rounded corners] (-9, -6) rectangle (-3.8, 1.5) {};
  \node[align=center] at (-0.6,1) {System B};
  \draw[rounded corners] (-3.2, -6) rectangle (2, 1.5) {};
  \seqnode{B_cpu}{RAM};
  \seqnode[left of=B_cpu]{B_nic}{NIC};
  \hseqnode[right of=B_cpu, node distance=1.5cm]{B_acpu}{};
  \seqnode[left of=B_cpu, node distance=7cm]{A_cpu}{CPU / RAM};
  \seqnode[right of=A_cpu]{A_nic}{NIC};
  \node[align=center, circle, draw=black, minimum size=.5mm] at (-5.7,-0.7) {\small 1};
  \msg{A_cpu}{A_nic}{.25}{WR MMIO}
  \msg[below]{A_cpu}{A_nic}{.3}{payload DMA}
  \msg{A_nic}{B_nic}{.5}{network transfer}
  \node[align=center, circle, draw=black, minimum size=.5mm] at (-4.5,-2) {\small 2};
  \msg{B_nic}{B_cpu}{.65}{payload DMA}
  \node[align=center, circle, draw=black, minimum size=.5mm] at (-1.3,-2.6) {\small 3};
  \msg[below]{B_nic}{A_nic}{.71}{Acknowledgement}
  \node[align=center, circle, draw=black, minimum size=.5mm] at (-4.5,-4.3) {\small 4};
  \msg[below]{A_nic}{A_cpu}{.76}{DMA CQE}
\end{tikzpicture}
\end{center}
\caption{RDMA Write sequence}
\label{fig:seq-wrt}
\end{figure}

Figure~\ref{fig:seq-wrt} show the operations involved in writing data to the remote using RDMA write. It is generally very
similar to the send and receive sequence presented in the previous section. The main difference is that the remote system does
not need to post a receive buffer and there is no generated CQE at the remote. Also obviously we will have to provide the 
remote address to write to in the Work Request.

\paragraph{} There also exists a related operation called \emph{Write with Immediate}, which works very similar but has two
differences. We are able to send an additional 4 byte large \emph{Immediate} value that we can add to the Work Request. 
Further Write with Immediate will also consume a posted receive request, in the same way as the send verb. It will however 
not write anything in the associated receive buffer, but it will generate a CQE. This CQE will contain the immediate value.
This allows us to notify the remote that we have written data and to send additional out of band data, but it also negates 
any performance improvements over the send verb.


\subsubsection{Read}

\subsubsection{Atomics}

The \emph{Verbs API} also allows us to perform two different atomic operations on remote 64 bit values. These operations also
do not need any involvement by the remote CPU and the necessary operations to perform these are very similar to the read verb.


\begin{itemize}
  \item \textbf{Fetch and Add} will read a 64 bit value from the remote memory space, add the additionally provided 
    \code{compare\_add} value, and write the original data before the add to the provided local memory.
  \item With \textbf{Compare and Swap} will read a 64 bit value from the remote memory space and will compare it to 
    the \code{compare\_add} value provided in the work request. If they are equal it will write the value \code{swap}
    which is also provided in the work request. Regardless of success, it will return the original data to the requester.
\end{itemize}

Atomic operations open up a lot of options for more complex protocols, but they are currently rarely used as they generally
provide much lower throughput~\cite{anuj-guide}.


