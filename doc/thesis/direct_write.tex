\section{Direct Write}\label{sec:conn:direct_write}

\subsection{Design} \label{sendrcv-design}
The idea of the \emph{Unbuffered} or \emph{Direct Write Protocol} is to avoid unnecessary copying we have when using
a buffered write connection with a ring buffer. We can achieve this by having the receiver specify specific memory 
regions for the sender to write to. 

\paragraph{} For this thesis we implemented something very reminiscent of the send receive protocol where the receiver 
posts receive buffer by sending the corresponding metadata to the sender, which will use these buffers in order. This does
not really make total use of this protocol as one would still need to copy the received message out of these buffers. This
could however fairly easily be extended by sending additional information with receive buffer, allowing the sender to write
the message to the correct buffer.\comment{not sure if this is clear / how to correctly explain this}


\begin{figure}[!htb]
\begin{center}
\begin{tikzpicture}[node distance=2cm,auto,>=stealth', minimum width=.75cm,minimum height=.5cm]
  \draw[rounded corners] (-.5, -3) rectangle (-4.5, 3) {};
  \node[align=center] at (-2.5, 2.5) {Sender};
  \draw[rounded corners] (.5, -3) rectangle (4.5, 3) {};
  \node[align=center] at (2.5, 2.5) {Receiver};

  \node [fill=white] at (-2.5, 1.5) (A) {\tiny addr};
  \node [anchor=west, fill=white] at (A.east) (A1) {\tiny rkey};
  \node [anchor=west, minimum width=.25cm, fill=white] at (A1.east) (A2) {\tiny v};

  \node [anchor=north, fill=black!35] at (A.south) (B) {\tiny a3};
  \node [anchor=west, fill=black!35] at (B.east) (B1) {\tiny 1};
  \node [anchor=west, minimum width=.25cm, fill=black!35] at (B1.east) (B2) {\tiny 0};

  \node [anchor=north, fill=black!35] at (B.south) (C) {\tiny b5};
  \node [anchor=west, fill=black!35] at (C.east) (C1) {\tiny 1};
  \node [anchor=west, minimum width=.25cm, fill=black!35] at (C1.east) (C2) {\tiny 0};

  \node [anchor=north, fill=black!05] at (C.south) (D) {\tiny c8};
  \node [anchor=west, fill=black!05] at (D.east) (D1) {\tiny 1};
  \node [anchor=west, minimum width=.25cm, fill=black!05] at (D1.east) (D2) {\tiny 1};
  \draw [<-,black!40!blue] (D.west) --  +(-.5,0)
        node [black!40!blue,left,inner xsep=.2, draw=none] (Tail) {\tiny next send};


  \node [anchor=north, fill=black!05] at (D.south) (E) {\tiny d2};
  \node [anchor=west, fill=black!05] at (E.east) (E1) {\tiny 1};
  \node [anchor=west, minimum width=.25cm, fill=black!05] at (E1.east) (E2) {\tiny 1};

  \node [anchor=north, fill=black!35] at (E.south) (F) {\tiny f3};
  \node [anchor=west, fill=black!35] at (F.east) (F1) {\tiny 1};
  \node [anchor=west, minimum width=.25cm, fill=black!35] at (F1.east) (F2) {\tiny 0};
  \draw [<-,black!40!green] (F2.east) --  +(.5,0)
        node [black!40!green,right,inner xsep=.2, draw=none] (head) {\tiny next free};



  \node [fill=black!35, minimum width=2cm] at (2.75, 1.5) (rB) {};
  \node [anchor=east, fill=white, minimum width=.5cm] at (rB.west) (rB1) {\tiny a3};
  \node [anchor=west, fill=black!35, minimum width=.5cm] at (rB.east) (rB2) {\tiny 1};

  \node [anchor=north, fill=black!35, minimum width=2cm, outer ysep=2] at (rB.south) (rC) {};
  \node [anchor=east, fill=white, minimum width=.5cm] at (rC.west) (rC1) {\tiny b5};
  \node [anchor=west, fill=black!35, minimum width=.5cm] at (rC.east) (rC2) {\tiny 1};

  \node [anchor=north, fill=black!05, minimum width=2cm, outer ysep=2] at (rC.south) (rD) {};
  \node [anchor=east, fill=white, minimum width=.5cm] at (rD.west) (rD1) {\tiny c8};
  \node [anchor=west, fill=black!05, minimum width=.5cm] at (rD.east) (rD2) {\tiny 0};

  \node [anchor=north, fill=black!05, minimum width=2cm, outer ysep=2] at (rD.south) (rE) {};
  \node [anchor=east, fill=white, minimum width=.5cm] at (rE.west) (rE1) {\tiny d2};
  \node [anchor=west, fill=black!05, minimum width=.5cm] at (rE.east) (rE2) {\tiny 0};

  \node [anchor=north, fill=black!35, minimum width=2cm, outer ysep=2] at (rE.south) (rF) {};
  \node [anchor=east, fill=white, minimum width=.5cm] at (rF.west) (rF1) {\tiny f3};
  \node [anchor=west, fill=black!35, minimum width=.5cm] at (rF.east) (rF2) {\tiny 1};

\end{tikzpicture}
\end{center}
\caption{Resources of the Direct Send Connection}
\label{fig:dirwrite-resources}
\end{figure}

Figure~\ref{fig:dirwrite-resources} is a representation of the data structures involved in the direct write 
connection. In the following subsections we will walk tough the steps involved in sending and receiving a message and
will reference these data structures.

\subsubsection{Sender}

The sender has an array of structs, which we will call \code{BufferInfo}, in local memory. This array is populated by the 
receiver and represent receive buffer to which the sender is allowed to write to. The struct contains three fields: 
\code{addr}, which is the virtual address of the corresponding buffer in the receivers memory, \code{rkey}, which is the 
rkey of that buffer, and \code{valid} which indicates whether this entry is actually valid.

\paragraph{} This array acts as a ring buffer, so when sending a message, the sender will look at the next field. If it is
valid it will send the message to the corresponding buffer at the receiver and invalidates it. We will always send the complete
buffer followed by a signaling byte, which will allow the receiver to poll on this byte to notice incoming messages.

\paragraph{} Our implementation of the sender is very simple, but it can easily be extended to be more sophisticated by adding
tags to the \code{BufferInfo} or by having multiple different arrays, which would allows us to \emph{route} the message to the
correct buffer at the sender and potentially eliminate the need for a further copy at the receiver. One could also change
the way of signaling an incoming message by switching to write with immediate or by applying a metadata approach as implemented
for the buffered write protocol.

\subsubsection{Receiver}

In a way the direct write receiver works in a similar way to the send and receive receiver. It can \code{Receive()} and 
\code{Free()} buffers. 

\paragraph{}\emph{Freeing} a buffer means making it available for the sender to write to. As we established at the sender 
there is a queue of \code{BufferInfo}. To \emph{free} a buffer we will first set the very last signal byte to 0, then we
write its address, rkey, and a valid byte of 1 to the next empty or invalidated queue entry. The receiver will keep a very 
similar local queue to keep track of the oder of posted buffers.

\paragraph{} To \emph{Receive} a buffer the receiver polls on the signal bit of the oldest freed buffer. This will be set
to one when the transmission is completed. We are again using the fact that RDMA updates memory in increasing memory order, 
for any modern systems~\cite{herd, farm}.


\begin{figure}[htp]
\includegraphics[width=1\textwidth]{write-direct-lat-msgsize.png}
\label{fig:plot-wdir-lat}
\end{figure}


\begin{figure}[htp]
\includegraphics[width=1\textwidth]{write-direct-bw-msgsize.png}
\label{fig:plot-wdir-bw}
\end{figure}



\begin{figure}[htp]
\includegraphics[width=1\textwidth]{write-direct-bw-threads.png}
\label{fig:plot-wdir-bw-threads}
\end{figure}


\begin{figure}[htp]
\includegraphics[width=1\textwidth]{write-direct-bw-n1.png}
\label{fig:plot-wdir-bw-n1}
\end{figure}
