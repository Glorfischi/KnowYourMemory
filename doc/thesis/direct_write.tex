\section{Direct Write}\label{sec:conn:direct_write}

The idea of the \emph{Unbuffered} or \emph{Direct Write Protocol} is to avoid the unnecessary copying we have when using
a buffered write connection with a ring-buffer. We can achieve this by having the receiver specify  memory 
regions for the sender to write to. 

\paragraph{} For this thesis we implemented something very reminiscent of the send receive protocol. The core idea is for
the receiver to send information on prepared receive buffers to the sender. The sender will then use these buffers in order. 



\begin{figure}[!htb]
\begin{center}
\begin{tikzpicture}[node distance=2cm,auto,>=stealth', minimum width=.75cm,minimum height=.5cm]
  \draw[rounded corners] (-.5, -3) rectangle (-4.5, 3) {};
  \node[align=center] at (-2.5, 2.5) {Sender};
  \draw[rounded corners] (.5, -3) rectangle (4.5, 3) {};
  \node[align=center] at (2.5, 2.5) {Receiver};

  \node [fill=white] at (-2.5, 1.5) (A) {\tiny addr};
  \node [anchor=west, minimum width=.25cm, fill=white] at (A.east) (A2) {\tiny v};

  \node [anchor=north, fill=black!35] at (A.south) (B) {\tiny a3};
  \node [anchor=west, minimum width=.25cm, fill=black!35] at (B.east) (B2) {\tiny 0};

  \node [anchor=north, fill=black!35] at (B.south) (C) {\tiny b5};
  \node [anchor=west, minimum width=.25cm, fill=black!35] at (C.east) (C2) {\tiny 0};

  \node [anchor=north, fill=black!05] at (C.south) (D) {\tiny c8};
  \node [anchor=west, minimum width=.25cm, fill=black!05] at (D.east) (D2) {\tiny 1};
  \draw [<-,black!40!blue] (D.west) --  +(-.5,0)
        node [black!40!blue,left,inner xsep=.2, draw=none] (Tail) {\tiny next send};


  \node [anchor=north, fill=black!05] at (D.south) (E) {\tiny d2};
  \node [anchor=west, minimum width=.25cm, fill=black!05] at (E.east) (E2) {\tiny 1};

  \node [anchor=north, fill=black!35] at (E.south) (F) {\tiny f3};
  \node [anchor=west, minimum width=.25cm, fill=black!35] at (F.east) (F2) {\tiny 0};
  \draw [<-,black!40!green] (F2.east) --  +(.5,0)
        node [black!40!green,right,inner xsep=.2, draw=none] (head) {\tiny next free};



  \node [fill=black!35, minimum width=2cm] at (2.75, 1.5) (rB) {};
  \node [anchor=east, fill=white, minimum width=.5cm] at (rB.west) (rB1) {\tiny a3};
  \node [anchor=west, fill=black!35, minimum width=.5cm] at (rB.east) (rB2) {\tiny 1};

  \node [anchor=north, fill=black!35, minimum width=2cm, outer ysep=2] at (rB.south) (rC) {};
  \node [anchor=east, fill=white, minimum width=.5cm] at (rC.west) (rC1) {\tiny b5};
  \node [anchor=west, fill=black!35, minimum width=.5cm] at (rC.east) (rC2) {\tiny 1};

  \node [anchor=north, fill=black!05, minimum width=2cm, outer ysep=2] at (rC.south) (rD) {};
  \node [anchor=east, fill=white, minimum width=.5cm] at (rD.west) (rD1) {\tiny c8};
  \node [anchor=west, fill=black!05, minimum width=.5cm] at (rD.east) (rD2) {\tiny 0};

  \node [anchor=north, fill=black!05, minimum width=2cm, outer ysep=2] at (rD.south) (rE) {};
  \node [anchor=east, fill=white, minimum width=.5cm] at (rE.west) (rE1) {\tiny d2};
  \node [anchor=west, fill=black!05, minimum width=.5cm] at (rE.east) (rE2) {\tiny 0};

  \node [anchor=north, fill=black!35, minimum width=2cm, outer ysep=2] at (rE.south) (rF) {};
  \node [anchor=east, fill=white, minimum width=.5cm] at (rF.west) (rF1) {\tiny f3};
  \node [anchor=west, fill=black!35, minimum width=.5cm] at (rF.east) (rF2) {\tiny 1};

\end{tikzpicture}
\end{center}
\caption{Resources of the Direct Send Connection}
\label{fig:dirwrite-resources}
\end{figure}

Figure~\ref{fig:dirwrite-resources} is a representation of the data structures involved in the direct write 
connection. In the following subsections we will walk tough the steps involved in sending and receiving a message and
will reference these data structures.

\subsection{Protocol}

The sender has an array of structs, which we call \code{BufferInfo} in local memory. This array is populated by the 
receiver and represent receive buffers to which the sender is allowed to write to. The struct contains two fields: 
\code{addr}, which is the virtual address of the corresponding buffer in the receivers memory, and \code{valid} which 
indicates whether this entry is actually valid. 

\paragraph{} This array acts as a ring-buffer, so when sending a message, the sender looks at the next field. If it is
valid it sends the message to the corresponding buffer at the receiver and invalidates it. We always send the complete
buffer followed by a signaling byte, which allows the receiver to poll on this byte to notice incoming messages.

\paragraph{} Our implementation of the sender is very simple, but it can easily be extended to be more sophisticated by adding
tags to the \code{BufferInfo} or by having multiple different arrays, which would allows us to \emph{route} the message to the
correct buffer at the sender and potentially eliminate the need for a further copy at the receiver. One could also change
the way of signaling an incoming message by switching to write with immediate or by applying a metadata approach as implemented
for the buffered write protocol.


\paragraph{} In a way the direct write receiver works in a similar way to the send and receive receiver. It can \code{Receive()} and 
\code{Free()} buffers. 

\emph{Freeing} a buffer means making it available for the sender to write to. As we established  
there is a queue of \code{BufferInfo} in the senders memory. To \emph{free} a buffer the receiver first sets the very last 
signal byte of its local buffer to 0, then we write its address and a valid byte of 1 to the next empty or 
invalidated queue entry at the sender. The receiver keeps a very similar local queue to keep track of the oder of 
posted buffers.

To \emph{Receive} a buffer the receiver polls on the signal bit of the oldest freed buffer. This will be set
to one by the sender when the transmission is completed. We are using the fact that RDMA updates memory in increasing memory 
order, for any modern systems~\cite{herd, farm}. This guarantees us that the complete message has been written as soon a we
see an update to the last byte.

\subsection{Features Analysis}

With our Direct Write connection we essentially rebuilt the send and receive verbs using only RDMA writes. This gives us 
similar features but also allows for more control over the protocol. This could potentially allow us to extend it for 
specific systems.

\paragraph{} Our current implementation fulfills our requirements for being \emph{non-blocking}, the same way the send-receive
protocol does. It however does not provide any \emph{interrupts} and we did not explore any \emph{resource sharing} approaches.
As it stands it also does not provide \emph{True Zero Copy} capabilities and it lacks support for 
\emph{variable message sizes}.


\paragraph{} We do however think that this protocol can be adapted to enable more features. By adding metadata when posting a
buffer we could enable \emph{True Zero Copy} capabilities for certain applications, by performing some routing decisions at
the sender and being able to directly write to the correct final memory location.
And with a more sophisticated buffer management one could more effectively utilizes the available memory by supporting 
\emph{variable message sizes}.



