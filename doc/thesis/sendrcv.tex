\subsection{Send Receive}\label{sendrcv}
\subsubsection{Design} \label{sendrcv-design}
\todo{Description of the connection}




\begin{figure}[!ht]
\begin{center}
\begin{tikzpicture}[node distance=2cm,auto,>=stealth']
  \seqnode{B_cpu}{RAM};
  \seqnode[left of=B_cpu]{B_nic}{NIC};
  \hseqnode[right of=B_cpu, node distance=1.5cm]{B_acpu}{};
  \seqnode[left of=B_cpu, node distance=7cm]{A_cpu}{CPU / RAM};
  \seqnode[right of=A_cpu]{A_nic}{NIC};
  \msg{A_cpu}{A_nic}{.25}{WR MMIO}
  \msg[below]{A_cpu}{A_nic}{.3}{payload DMA}
  \msg{A_nic}{B_nic}{.5}{network transfer}
  \msg{B_nic}{B_cpu}{.65}{payload DMA}
  \msg[below]{B_nic}{B_cpu}{.7}{WQE DMA}
  \fetch{B_acpu}{B_cpu}{.8}{poll CQ}
\end{tikzpicture}
\end{center}
\caption{Send Receive sequence}
\label{fig:seq-sndrcv}
\end{figure}




\begin{enumerate}
  \item We start a transfer by copying a \emph{Work Request} to the \emph{NIC} using Memory Mapped IO (MMIO). This gives
    us constant overhead $o_{snd}$ independent of the size of the message.
  \item Given the information in the \emph{Work Request} the \emph{NIC} then accesses the messages payload using DMA. This 
    might generate more than one PCIe transaction \cite{atc16-kalia}. The payload is then sent over the network.
  \item As soon as the receiver receives the first segment it will consume posted receive buffer, write incoming payload 
    to the receive buffer and generate a \emph{Work Completion Event (WQE)}. We will call the combined overhead of consuming a receive
    buffer and generating a WQE $o_{rcv}$, which is also independent from $k$.
  \item At last the receiving CPU will have to poll the \emph{Completion Queue (CQ)}. This will again give us a constant 
    overhead of $o_{cq}$.
\end{enumerate}

\paragraph{Inline Sending}
\todo{Describe inline sending. How does our performance model change?}
\todo{What about inline receiving?}
\paragraph{Shared Receive Queues}
\todo{Look at srq. Talk about congestion when reposting buffers?}

\subsubsection{Evaluation}
\todo{A lot of plots? Should this be a separate subsection or should we just scatter evaluation throughout the section?}


