\section{Performance Model}\label{sec:perf-model} \label{sec:model}

We looked at what is involved in an RDMA transmission in Figure \ref{fig:seq-sndrcv}.\todo{Figure in RDMA background}
Trying to closely model this however gives us a far too complex model, with a lot of parameters that are hard to assess,
especially if we start to look at more complex protocols. 

We simplify our model by using a modified \emph{LogGP model}~\cite{}. The LogGP model was designed to model point to point
communication with variable message sizes. \todo{Intro to LogGP}


The LogGP model was designed for classical IP based messaging and did not take into account any offloading to the NIC. We 
can however integrate the heavy offloading happening in RDMA by splitting the offset $o$ into multiple offsets for each
of the components. This leaves us with the following parameters.


\begin{itemize}
  \item $L$: an upper bound on the Latency, incurred in sending a message from the senders NIC to the receivers NIC
  \item $G$: the Gap per byte for long messages. For our purposes this is the time of sending a single byte given the 
    maximum bandwidth of our link.
  \item $g$: he gap, defined as the minimum time interval between consecutive message transmissions.
  \item $P$: the number of processes (or servers).
  \item $o_{snd}$: the \emph{send overhead}, defined as the length of time that a processor is engaged in sending each message.
  \item $o_{nsnd}$: the \emph{send NIC overhead}, defined as the length of time that a NIC is engaged in sending each message.
  \item $o_{rcv}$: the \emph{receive overhead}, defined as the length of time that a processor is engaged in receiving each message.
  \item $o_{nrcv}$: the \emph{receive NIC overhead}, defined as the length of time that a NIC is engaged in receiving each message.
  \item $g_{rcv}$: the \emph{receive gap}, defined as the time interval between the receiving processor having received a message 
    and it being ready to process the next message. One example of this gap would be preparing a new receive buffer.
\end{itemize}

\begin{figure}[!ht]
\begin{center}
\begin{tikzpicture}[node distance=1cm,auto,>=stealth']
  \node[] (p0) {P0};
  \node[right of=p0, node distance=10cm] (p0_g) {};
  \draw[dotted] (p0) -- (p0_g);

  \node[below of=p0, node distance=0.5cm] (p0_nic) {P0 NIC};
  \node[right of=p0_nic, node distance=10cm] (p0_nic_g) {};
  \draw[dotted] (p0_nic) -- (p0_nic_g);

  \node[below of=p0_nic, node distance=1.5cm] (p1_nic) {P1 NIC};
  \node[right of=p1_nic, node distance=10cm] (p1_nic_g) {};
  \draw[dotted] (p1_nic) -- (p1_nic_g);

  \node[below of=p1_nic, node distance=0.5cm] (p1) {P1};
  \node[right of=p1, node distance=10cm] (p1_g) {};
  \draw[dotted] (p1) -- (p1_g);
  

  %%%%

  \draw[very thick] (p0) --node[above,scale=0.75,midway]{$o_{snd}$} ($(p0)!0.15!(p0_g)$);
  \draw[very thick] ($(p0_nic)!0.15!(p0_nic_g)$) --node[above,scale=0.75,midway]{$o_{nsnd}$} ($(p0_nic)!0.225!(p0_nic_g)$);


  \path[] ($(p0_nic)!0.225!(p0_nic_g)$) --node[above,scale=0.75,midway]{$G$} ($(p0_nic)!0.26!(p0_nic_g)$);
  \draw[dotted, ->] ($(p0_nic)!0.225!(p0_nic_g)$) -- ($(p1_nic)!0.395!(p1_nic_g)$);

  \path[] ($(p0_nic)!0.26!(p0_nic_g)$) --node[above,scale=0.75,midway]{$G$} ($(p0_nic)!0.295!(p0_nic_g)$);
  \draw[dotted, ->] ($(p0_nic)!0.26!(p0_nic_g)$) -- ($(p1_nic)!0.43!(p1_nic_g)$);

  \path[] ($(p0_nic)!0.295!(p0_nic_g)$) --node[above,scale=0.75,midway]{$G$} ($(p0_nic)!0.33!(p0_nic_g)$);
  \draw[dotted, ->] ($(p0_nic)!0.295!(p0_nic_g)$) -- ($(p1_nic)!0.465!(p1_nic_g)$);

  \draw[dotted, ->] ($(p0_nic)!0.33!(p0_nic_g)$) -- ($(p1_nic)!0.5!(p1_nic_g)$);


  \draw[very thick] ($(p1_nic)!0.5!(p1_nic_g)$) --node[above,scale=0.75,midway]{$o_{nrcv}$} ($(p1_nic)!0.55!(p1_nic_g)$);
  \draw[very thick] ($(p1)!0.55!(p1_g)$) --node[above,scale=0.75,midway]{$o_{rcv}$} ($(p1)!0.6!(p1_g)$);
    
  %%%%

  \draw[dotted] ($(p0_nic)!0.225!(p0_nic_g)$) -- ($(p0_nic)!0.225!(p0_nic_g)+(0,-3)$);
  \draw[dotted] ($(p0_nic)!0.330!(p0_nic_g)$) -- ($(p0_nic)!0.330!(p0_nic_g)+(0,-3)$);
  \draw[<->] ($(p0_nic)!0.225!(p0_nic_g)+(0,-2.8)$) --node[above,scale=0.75,midway]{$(k-1)G$} ($(p0_nic)!0.330!(p0_nic_g)+(0,-2.8)$);

  \draw[dotted] ($(p1_nic)!0.5!(p1_nic_g)$) -- ($(p1_nic)!0.5!(p1_nic_g)+(0,-1.5)$);
  \draw[<->] ($(p0_nic)!0.330!(p0_nic_g)+(0,-2.8)$) --node[above,scale=0.75,midway]{$L$} ($(p1_nic)!0.5!(p1_nic_g)+(0,-1.3)$);

  \draw[dotted] ($(p1)!0.6!(p1_g)$) -- ($(p1)!0.6!(p1_g)+(0,-1)$);
  \draw[dotted] ($(p1)!0.65!(p1_g)$) -- ($(p1)!0.65!(p1_g)+(0,-1)$);
  \draw[<->] ($(p1)!0.6!(p1_g)+(0,-.8)$) --node[above,scale=0.75,midway]{$g_{rcv}$} ($(p1)!0.65!(p1_g)+(0,-.8)$);

  \draw[dotted] ($(p0)!0.15!(p0_g)$) -- ($(p0)!0.15!(p0_g)+(0,.8)$);
  \draw[dotted] ($(p0)!0.35!(p0_g)$) -- ($(p0)!0.35!(p0_g)+(0,.8)$);
  \draw[<->] ($(p0)!0.15!(p0_g)+(0,.6)$) --node[above,scale=0.75,midway]{$g$} ($(p0)!0.35!(p0_g)+(0,.6)$);

  %%%%

  \draw[very thick, lightgray] ($(p0)!0.2!(p0_g)$) --node[above,scale=0.75,midway]{$o_{snd}$} ($(p0)!0.35!(p0_g)$);
  \draw[very thick, lightgray] ($(p0_nic)!0.35!(p0_nic_g)$) --node[above,scale=0.75,midway]{$o_{nsnd}$} ($(p0_nic)!0.425!(p0_nic_g)$);

  \path[lightgray] ($(p0_nic)!0.425!(p0_nic_g)$) --node[above,scale=0.75,midway]{$G$} ($(p0_nic)!0.45!(p0_nic_g)$);
  \draw[dotted, ->, lightgray] ($(p0_nic)!0.425!(p0_nic_g)$) -- ($(p1_nic)!0.625!(p1_nic_g)$);

  \path[lightgray] ($(p0_nic)!0.45!(p0_nic_g)$) --node[above,scale=0.75,midway]{$G$} ($(p0_nic)!0.475!(p0_nic_g)$);
  \draw[dotted, ->, lightgray] ($(p0_nic)!0.45!(p0_nic_g)$) -- ($(p1_nic)!0.65!(p1_nic_g)$);

  \path[lightgray] ($(p0_nic)!0.475!(p0_nic_g)$) --node[above,scale=0.75,midway]{$G$} ($(p0_nic)!0.5!(p0_nic_g)$);
  \draw[dotted, ->, lightgray] ($(p0_nic)!0.475!(p0_nic_g)$) -- ($(p1_nic)!0.675!(p1_nic_g)$);

  \draw[dotted, ->, lightgray] ($(p0_nic)!0.5!(p0_nic_g)$) -- ($(p1_nic)!0.7!(p1_nic_g)$);


  \draw[very thick, lightgray] ($(p1_nic)!0.7!(p1_nic_g)$) --node[above,scale=0.75,midway]{$o_{nrcv}$} ($(p1_nic)!0.75!(p1_nic_g)$);
  \draw[very thick, lightgray] ($(p1)!0.75!(p1_g)$) --node[above,scale=0.75,midway]{$o_{rcv}$} ($(p1)!0.8!(p1_g)$);



\end{tikzpicture}
\end{center}
\caption{Sending an receiving messages under our model}
\label{fig:model-base}
\end{figure}


\paragraph{Latency Estimate}

Using this model we can estimate the latency $t$ of transferring a single message $m$ of size $k$ with:

$$
t \geq o_{snd} + o_{nsnd}  + (k-1)G + L + o_{nrcv} + o_{rcv}
$$


\paragraph{Throughput Estimate}

Message transfer is highly pipelined. So to estimate bandwidth $bw$ our model basically reduces to finding the bottleneck.

$$
bw \leq \max ( \frac{k}{o_{snd} + g_{snd}}, \frac{k}{o_{nsnd} + (k-1)G}, \frac{k}{o_{nrcv} + (k-1)G}, \frac{k}{o_{rcv} + g_{rcv}})
$$
