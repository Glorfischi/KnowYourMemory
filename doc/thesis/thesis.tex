\documentclass{article}

\usepackage{pdfpages}
\usepackage[nottoc]{tocbibind}
\usepackage{color}
\usepackage{wrapfig}
\usepackage{graphicx}
\usepackage{subcaption}
\usepackage{float}
\usepackage{wrapfig}
\graphicspath{ {../plots/} }
\usepackage{tikz}
\usetikzlibrary{calc,positioning,arrows}
\usetikzlibrary{shapes.misc, positioning}
\usetikzlibrary{patterns}
\usetikzlibrary{decorations.text}
\usetikzlibrary{shapes}
\usepackage{amsmath,amsthm}

\usepackage{babel}

\usepackage{listings}
\usepackage{color}

\iffalse
\newcommand{\comment}[1]{\textit{\textcolor{blue}{/* #1 */} }}
\fi
\usepackage{marginnote}
\newcommand{\todo}[1]{\textit{\textcolor{red}{// TODO: #1} }}
\newcommand{\comment}[1]{\textcolor{red}{*}\marginnote{\textit{\textcolor{red}{#1} }}}
\newcommand{\draft}[1]{\textit{\textcolor{olive}{#1} }}
\newcommand{\code}[1]{\texttt{#1}}


\theoremstyle{plain}
\newtheorem{thm}{Theorem}[section]
\newtheorem{lem}[thm]{Lemma}
\newtheorem{prop}[thm]{Proposition}
\newtheorem*{cor}{Corollary}

\theoremstyle{definition}
\newtheorem{defn}{Definition}[section]
\newtheorem{conj}{Conjecture}[section]
\newtheorem{exmp}{Example}[section]

\theoremstyle{remark}
\newtheorem*{rem}{Remark}
\newtheorem*{note}{Note}


% For tables
\usepackage{etoolbox}
\usepackage{array}
\usepackage{threeparttable}
\usepackage{colortbl}

\newcolumntype{x}[1]{>{\centering\let\newline\\\arraybackslash}p{#1}}
 
\definecolor{myc}{rgb}{1,0.88,0.88}



\begin{document}
\iffalse
\fi
%% Helper function for squence diagrams
\newcommand{\seqnode}[3][]{ 
  \node[#1] (#2) {#3};
  \node[below of=#2, node distance=5cm] (#2_g) {};
  \draw (#2) -- (#2_g);
}
\newcommand{\hseqnode}[3][]{ 
  \node[#1] (#2) {#3};
  \node[below of=#2, node distance=5cm] (#2_g) {};
}
\newcommand{\msg}[5][above]{
  \draw[->] ($(#2)!#4!(#2_g)$) -- node[#1,scale=0.75,midway]{#5} ($(#3)!#4+0.04!(#3_g)$);
}
\newcommand{\fetch}[4]{
  \draw[-] ($(#1)!#3-0.04!(#1_g)$) -- node[above,scale=0.75,midway]{#4} ($(#2)!#3!(#2_g)$);
  \draw[->] ($(#2)!#3!(#2_g)$) -- node[above,scale=0.75,midway]{} ($(#1)!#3+0.04!(#1_g)$);
}
%% Helper functions for graphics

\newcommand{\queue}[2][]{ 
  \draw (#2) -- ++(2cm,0) -- ++(0,-1cm) -- ++(-2cm,0);
  \foreach \i in {1,...,4} \draw (#2)++(2cm-\i*3mm,0) -- +(0,-1cm);
  \node[align=center, , xshift = 1cm, yshift = .15cm] at (#2) {#1};
}

\newcommand{\rqueue}[2][]{ 
  \draw (#2) -- ++(2cm,0);
  \draw (#2) -- ++(0,-1cm) -- ++(2cm,0);
  \foreach \i in {1,...,4} \draw (#2)++(\i*3mm,0) -- +(0,-1cm);
  \node[align=center, , xshift = 1cm, yshift = .15cm] at (#2) {#1};
}


\tableofcontents
\pagebreak
\section{Introduction}

Remote Direct Memory Access (RMDA) is a powerful communication mechanism that offers the potential for exceptional performance.
RDMA allows one machine to directly access the memory of a remote machine across the network without the interaction of the 
remote CPU. This gives developers a plethora of options to implement communication protocols. However using these options 
effectively is nontrivial and the observed performance can vary greatly for seemingly minor differences. 

Existing research either primarily focus on evaluating very low level verb performance~\cite{anuj-guide} or focus strongly on 
Remote Procedure Calls (RPCs)~\cite{eval-mpp} often comparing the observed performance to using remote data 
structures~\cite{fasst, rpc-vs-rdma}. Nearly all of them employ naive message passing protocols using either 
send receive or RDMA writes with ring-buffers~\cite{rdma-fast-dbms} or \emph{mailboxes}~\cite{ziegler2020rdma} and do not 
take full advantage of 
features offered by modern RDMA-capable network controllers. Further hardly any work looks into the usage of shared receive
queues, memory fences, or atomics for resource sharing.

\paragraph{}In this thesis we implement and evaluate various different message passing protocols. We show that there are a 
lot of ways to implement data exchange connections using less used RDMA features such as \emph{shared receive queues}, \emph{reads}, 
\emph{memory fences}, and \emph{RDMA atomics}. We also show that even common approaches such as ring-buffer based protocols 
can be implement in multiple ways giving us different performance characteristics and features.

\begin{itemize}
  \item We focus on implementing message passing protocols, without limiting us to RPCs. We believe this gives engineers 
    building blocks to develop more sophisticated protocols without micro-benchmarking basic verbs.
  \item We define other connection features outside of raw performance which have been relevant for applications such as 
    efficient resource usage.
  \item We implement and evaluate different message passing protocols. We reason why we explicitly implemented these protocols
    and evaluate them for different communication patterns.
  \item We provide a performance model to better understand the observed performance.
\end{itemize}

\paragraph{}We think that this work, with 
implementations of more unusual communication protocols, inspires system designers to develop 
new protocols that make more extensive use of modern RNICs.





\pagebreak
\section{Related Work}

\subsection{Using RDMA effectively}
\todo{Papers which make verb leve benchmarking. E.g. Anujs guidelines}


\todo{Papers looking at RPC, remote data strucutres, or socket like interfaces}


\subsection{RDMA Systems}
\todo{Papers presenting systems using rdma protocols}

\pagebreak
\section{RDMA}

Remote Direct Memory Access (RDMA) is a network feature that enables direct access to the memory of a remote computer, 
without any interaction by the remote CPU. The complete bypassing of the hosts kernel and even CPU allows us to achieve 
very low latency and high bandwidth, while reducing or eliminating CPU utilization.

While initially developed as part of the \emph{InfiniBand}~\cite{} network protocol used primarily in high performance computing,
it is also available for commodity ethernet using \emph{Internet Wide Area RDMA (iWARP)}~\cite{} or
\emph{RDMA over Converged Ethernet (RoCE)}~\cite{}. For the rest of the thesis we will focus on RoCE which is designed for 
intra data center communication and seems to be adopted the most in modern data centers. \comment{such claims need references}
While we focus on and will only evaluate RoCE, most of this work should be applicable to the other technologies.



\subsection{Verbs API}

All three technologies RoCE, iWARP, and InfiniBand share a common user API called \emph{Verbs API}. The verbs API gives us a
userspace library called \emph{libibverbs} which gives developers direct low level access to the device, bypassing the kernel.

\paragraph{}The verbs API is different from traditional socket programming. Applications interact directly with the Network 
Interface Card (NIC) through so called \emph{Queue Pairs (QP)} and \emph{Completion Queues (CQ)}, allowing it to issue
so called \emph{Verbs}, different operations that the NIC can perform. The verbs are:

\begin{itemize}
  \item Send (with Immidiate): Sends a data from the sender memory to a prepared memory region at the receiver
  \item Receive: Prepares a memory region to receive data through the send verb
  \item Write (with Immidiate): Copies data from the sender memory to known memory location at the receiver without any 
    inteaction from the remote CPU.
  \item Read: Copies data from remote memory to a local buffer without any inteaction from the remote CPU.
  \item Atomics: Two different atomic operations. Copare and Swap (CAS) and Fetch and Add (FAA). They can access 64-bit 
    values in the remote memory. 
\end{itemize}



\begin{figure}[!ht]
\begin{center}
\begin{tikzpicture}[node distance=2cm,auto,>=stealth']
  \queue[Send Queue]{0,3};
  \queue[Receive Queue]{0,1.5};
  \rqueue[Completion Queue]{0,0};

  \draw[rounded corners] (-.5, -1.5) rectangle (5.5, 3.5) {};
  \node[align=center] at (3cm, 3.2cm) {RNIC};
  \draw[rounded corners] (3.4, -.5) rectangle (5.1, 2.5) {};
  \node[align=center] at (4.25cm, 1cm) {Processing \\ Unit};
\end{tikzpicture}
\end{center}
\caption{Resources of the Verbs API}
\label{fig:rdma-parts}
\end{figure}


\paragraph{} A QP consists of two queues that are responsible to schedule work for the NIC. The \emph{Send Queue} and the \emph{Receive 
Queue}. A Work Request (WR) consists of a \code{opcode} which signifies which verb we want to execute, all necessary 
information to complete this operations, and contains a user provided Work Request ID \code{wr\_id}. As soon as the NIC has 
processed and complete the issued work Request it enqueues a Completion Queue Event (CQE) into an other queue called the
Completion Queue (CQ). The CQE will contain the user provided \code{wr\_id} and allows us to notice when a request was 
completed.\comment{Say something on in order guarantee?}


\paragraph{} It is worth noting that every buffer that is accessed by the NIC needs to be previously registered as a usable
\emph{Memory Region (MR)}.


\subsubsection{Send / Receive}
The \emph{Send} and \emph{Receive} verbs are the most traditional operations, which allows us to send a single message to 
a receiver. Let's walk through sending a message.

\begin{figure}[!ht]
\begin{center}
\begin{tikzpicture}[node distance=2cm,auto,>=stealth']
  \node[align=center] at (-6.4,1) {System A};
  \draw[rounded corners] (-9, -6) rectangle (-3.8, 1.5) {};
  \node[align=center] at (-0.6,1) {System B};
  \draw[rounded corners] (-3.2, -6) rectangle (2, 1.5) {};
  \seqnode{B_cpu}{RAM};
  \seqnode[left of=B_cpu]{B_nic}{NIC};
  \hseqnode[right of=B_cpu, node distance=1.5cm]{B_acpu}{};
  \seqnode[left of=B_cpu, node distance=7cm]{A_cpu}{CPU / RAM};
  \seqnode[right of=A_cpu]{A_nic}{NIC};
  \node[align=center, circle, draw=black, minimum size=.5mm] at (-1,-0.4) {\small 1};
  \msg{B_cpu}{B_nic}{.2}{WR MMIO}
  \node[align=center, circle, draw=black, minimum size=.5mm] at (-5.7,-0.7) {\small 2};
  \msg{A_cpu}{A_nic}{.25}{WR MMIO}
  \msg[below]{A_cpu}{A_nic}{.3}{payload DMA}
  \msg{A_nic}{B_nic}{.5}{network transfer}
  \node[align=center, circle, draw=black, minimum size=.5mm] at (-4.5,-2) {\small 3};
  \msg{B_nic}{B_cpu}{.65}{payload DMA}
  \msg[below]{B_nic}{B_cpu}{.7}{CQE DMA}
  \node[align=center, circle, draw=black, minimum size=.5mm] at (-1.3,-2.6) {\small 4};
  \msg[below]{B_nic}{A_nic}{.71}{Acknowledgement}
  \node[align=center, circle, draw=black, minimum size=.5mm] at (-4.5,-4.3) {\small 5};
  \msg[below]{A_nic}{A_cpu}{.76}{DMA CQE}
  \node[align=center, circle, draw=black, minimum size=.5mm] at (0.5,-4.5) {\small 6};
  \fetch{B_acpu}{B_cpu}{.8}{poll CQ}
\end{tikzpicture}
\end{center}
\caption{Send Receive sequence}
\label{fig:seq-sndrcv}
\end{figure}


Let's assume that that system A and B have set up a connection. Each of them have setup a QP and associated a Completion 
Queue to it. Both systems have registered a MR of at least the size of the to be sent message.

\begin{enumerate}
  \item First system B will have to post a \emph{Receive Buffer}, meaning it has to reserve a buffer for incoming messages.
    It does this by moving a Work Request into its Receive Queue. This WR will contain a pointer to the MR he prepared. We
    call this posting a receive buffer. System B will now poll its CQ until it receives a CQE for its issued receive request.
  \item Now system A will initiate the transfer by posting a \emph{Send Request}. It copies a Work Request to the Send 
    Queue, which contains the \code{IBV\_WR\_SEND} opcode and a pointer to its local buffer containing the to be send message
    and its size. It will then also start polling its CQ to notice the completion of the send request.
  \item The previous step was actually an MMIO operation, writing the Work Request to the NIC, which will now 
    accesses the messages payload using DMA. This  might generate more than one PCIe transaction \cite{atc16-kalia}. 
    The payload is then sent over the network.
  \item As soon as the receiver receives the first segment it will consume the posted receive buffer, write incoming payload 
    to the receive buffer and generate a \emph{Work Completion Event (WQE)}. 
  \item The successful writing of the message will generate an acknowledgement, which will be sent back to the sender where 
    its NIC will generate a CQE for the send request, which the sending CPU will be able to poll.
  \item At last the receiving CPU will be able to poll its \emph{Completion Queue (CQ)} and the transfer is complete.
\end{enumerate}


\todo{Outro. Maybe say something to Send with Immidiate?}






\subsubsection{Write}
\subsubsection{Read}
\subsubsection{Atomics}




\pagebreak
\section{Data Exchange Protocols}\label{sec:protocols}
The goal of this work is to design, implement, and study a comprehensive list of RDMA based protocols for data exchange. 
However, without a clear definition of \emph{data exchange} the number of possible protocols is virtually endless.

In this section we limit our definition of \emph{data exchange protocol} to that of a more clearly defined message passing
protocol, which reduces the size of our design space and allows us to compare protocols. We then introduce
features outside of raw performance that have been relevant for real-world applications. Finally, we introduce six 
protocols that all use different RDMA features and design approaches to implement message passing protocols with different 
capabilities and performance guarantees.

\subsection{Definition}\label{sec:proto-def}

Most research focuses on RPCs \cite{anuj-guide, fasst, herd}, remote datastructure access \cite{pilaf, farm}, or replications of
socket like interfaces \cite{socksdirect}. In this thesis we look at message passing protocols with fairly few guarantees.
We believe with this more general communication model we can give engineers and researchers a better understanding of the 
building blocks, that can be  used to build more specific connection protocols, like RPCs, while not only 
micro-benchmarking RDMA verbs.


\begin{defn}
We define a message passing protocol $P$ as an algorithm for moving a message $m$ that resides in the memory of
node $N_a$ to another node $N_b$. The transfer must be initiated by $N_a$. After the transfer, $N_b$ must know that the 
message was fully received and $N_a$ must know that it can reuse local resources that where used to send $m$.
\end{defn}

We explicitly have no synchronisation requirements, so the sender does not need to know whether the receiver has actually 
received the message.



\pagebreak

\subsection{Features} \label{sec:features} 


There is more to data exchange protocols than raw throughput or latency. Sometimes, it is more important for an application
that the connection is \emph{non-blocking} or that the memory requirements do not grow too large, even if that protocol wastes
a few microseconds.

To be able to better compare our presented protocols, we define a few traits that are often required by systems and
analyse which protocols provide these traits.

\paragraph{True Zero-Copy} A protocol is truly zero copy if it can receive directly to the destination address. While RDMA 
claims to be zero-copy, most protocols are not truly zero-copy. Protocols using ring-buffers or mailboxes are
strictly speaking not zero-copy, as the receiver will always have to copy the data from the buffer to its actual destination.
This is can be especially important when, for example, transferring a large amount of data that does not have to be further processed.

\paragraph{Variable Message Size} The protocol allows us to send messages of different sizes, without using the 
complete buffer at the receiver. Ring-buffers, for example, can be designed to only use the necessary space per message
while send receive will always use the complete receive buffer, no matter how small the message actually is. Variable
message sizes allow us to avoid memory fragmentation and reduce total memory usage in general.

\paragraph{Passive Sender/Receiver} It describes whether the sender or receiver is \emph{passive}. We define \emph{passivity} as not 
requiring any CPU involvement in the actual data transfer. That means a \emph{passive receiver} can receive messages by
simply reading from a memory location, without having to acknowledge anything or post receive buffers. \emph{Passivity} 
reduces the necessary CPU usage to a minimum and can be useful for applications with heterogeneous CPU requirements.

\paragraph{Interrupts} There exists some kind of notification system, that allows the receiver to be notified 
of an incoming message without having to constantly poll for it. While polling gives us better performance in almost all cases,
constantly polling wastes a lot of CPU cycles when we do not receive a lot of messages.

\paragraph{Resource Sharing} Multiple connections can share resources, especially memory. We are using \emph{reliable connections}
so we inherently need to use one \emph{QP} per sender and receiver pair, but we can reduce memory usage by sharing receive 
buffers between multiple connections. Without any memory sharing, utilization grows linearly with the number of QPs, which 
can quickly lead to unacceptable memory requirements, especially for $N$ to $N$ communication patterns.

\paragraph{Non-Blocking} By non-blocking we mean that a single not processed message cannot block a complete connection. 
It is very common in systems to distribute incoming messages to different threads. It is important that a single slow running 
task cannot completely block the system from making progress. The way we use it a ring-buffer is a good example of a blocking
behaviour. If a single buffer segment is not marked as available to be reused it may block the whole buffer. A send receive 
based connection however is non-blocking. As long as there is at least one posted receive buffer, the connection is able to
make progress.




\subsection{Design Space} \label{sec:proto-ds}


Even with the more limited definition of \emph{data exchange} in Section~\ref{sec:proto-def} there are countless numbers
of non-trivial ways to implement such a protocol using RDMA features. In this section we take a look at the design 
space of data exchange protocols, show different approaches, and point out specific protocols that we implemented
as an inspiration for other developers to design their own protocols.


\begin{table}[!ht]
\renewcommand{\arraystretch}{2}
\setlength\tabcolsep{1.5pt}
\centering
 \begin{tabular}{|x{2.4cm}|x{1.5cm}|x{1.5cm}|x{1.5cm}|x{1.5cm}|x{1.5cm}|x{1.5cm}|} % I specify the sizes of columns. x is for centering and p is  for left
 \hline
 Protocol    & Zero Copy & Variable Message Size & Passive Sender / Receiver & Interrupts & Non-Blocking & Resource Sharing\\
  \hline
  \hline
 Send-Receive (SR)   & & & & \checkmark  &  \checkmark & \checkmark\\
  \hline
  \hline
 Direct-Write (DW)   & (\checkmark) & & &  &  \checkmark & \\
  \hline
 Buffered-Write (BW)  &  & \checkmark & Receiver & (\checkmark) &   & \\
  \hline
 Shared-Write (SW)    &  & \checkmark &  & \checkmark &   & \checkmark \\
  \hline
  \hline
 Direct-Read (DR)    & \checkmark  & \checkmark &  & \checkmark &   \checkmark & (\checkmark)\\
  \hline
 Buffered-Read (BR)  &  & \checkmark & Sender &  &   & \\
\hline
\end{tabular}
\caption{Protocol overview and summary of their features}
\label{tab:protocols}
\end{table}


\paragraph{} One way to split all possible exchange protocols is by the verbs they use to transfer the actual 
data. We look at each of these verbs separately, as they all work in different ways and have different requirements.

We did not see any advantages in splitting the actual data transfer into multiple verbs, as this only seems to add overhead 
and complexity. And while it would theoretically be possible to use \emph{Compare and Swap} for generic data transfer,
the expected performance is far too low to get a viable protocol. So we do not look into these options but focus on the 
following verbs.

\paragraph{} We want to point out that this is in no way an exhausting list of protocols, but should rather show that the 
design space is very large and that there are a lot of interesting protocols that can still be explored.

\subsubsection{Send-Based Protocols}
The \emph{Send} verb basically fulfills our requirements for a \emph{data exchange protocol} out of the box. It is sender
initiated, the sender can reuse the buffer after the send is complete, and the receiver is notified of a new message by
a completion event.

There are, however, still multiple ways to tweak and tune a send-receive-based protocol by batching, inline sending, or 
sharing receive buffers between multiple connections using a \emph{Shared Receive Queue}. We explore these implementation 
details and their performance characteristics in Section \ref{sec:conn:send}.

\subsubsection{Write-Based Protocols}
The \emph{Write} verb is a lot less restrictive. It allows the sender to write to an arbitrary 
location in the receiver's memory, without any interaction of the receiver's CPU. 
This also means we need to solve some of the 
problems which the \emph{Send} verb solved for us.

\begin{itemize}
  \item The sender needs to know where to send it to and if the receiver is ready to receive the data.

  \item The receiver needs to notice that it received data.
\end{itemize}

We look at and implement two general approaches for write-based protocols, a \emph{buffered} and an \emph{unbuffered} approach.

\paragraph{} The idea of a buffered read connection is to have a dedicated, structured buffer at the receiver. In most cases, 
this is a ring-buffer. This allows the sender to always know where to write to without any communication overhead. There 
are, however, downsides to using a buffer to transfer data. Most importantly the buffer is usually not the place the 
data should finally end up in, which forces us to copy the received data again.

A more detailed analysis of \emph{Buffered Write Protocols} with our implementation using a ring-buffer is in 
Section \ref{sec:conn:buf_write}. Further in Section \ref{sec:conn:shared_write},
we look at a way to share this ring-buffer between multiple connections using RDMA atomics.


\paragraph{} The  \emph{Unbuffered Write Protocol} avoids the additional copy which is usually necessary when 
using a \emph{Buffered Write Protocol}. The receiver should be able to choose the target location for each message. 
That means, there is a communication overhead for each message, as the sender either needs to query the receiver where to 
write to or the receiver needs to preemptively send locations to write to.
For a more detailed analysis of \emph{Unbuffered Write Protocols} see Section \ref{sec:conn:direct_write}.


\paragraph{}Each of these two write-based protocol approaches can again be implemented in many different ways which can
in turn have drastically different performance characteristics. 


\subsubsection{Read-Based Protocols}
The \emph{Read} verb is generally very similar to the \emph{Write} verbs. This time it allows the receiver to read from 
an arbitrary location in the sender's memory, without any interaction of the sender's CPU. Any protocol
using \emph{Read} as the data transfer verb needs to solve two problems:

\begin{itemize}
  \item The receiver needs to know that new data is ready and where to read it from.
  \item The sender needs to be notified when the data transfer is completed.
\end{itemize}

We implemented a buffered as well as an unbuffered approach to a read-based protocol. 


\paragraph{} The buffered read works similarly to the buffer write approach: This time the \emph{sender} has a structured buffer where the 
\emph{receiver} can read from. This reduces the communication overhead as the receiver always knows where to read the next 
message from. Section \ref{sec:conn:buf_read} contains a more detailed look at \emph{Buffered Read Protocols} together with 
our implementation.

\paragraph{} A \emph{Unbuffered Read Protocol} avoids the additional copy which is usually necessary when using a buffer. This means the 
sender needs to notify the receiver when a new message arrives and where to read it from.  This introduces a large
communication overhead but by being truly zero-copy this approach can be efficient, especially for large messages.
We present our implementation of an \emph{Unbuffered Read Protocol} in Section~\ref{sec:conn:direct_read}.


\pagebreak
\subsection{Interface}

One of our goals is to be able to compare the different protocols we evaluate. We introduce a programming 
interface that is implemented by nearly all of the six presented connection types. We also describe a few implementation 
details that are shared by almost all of our connections.

\subsubsection{Sender}

Our protocol definition requires that the sender initiates the transfer. To achieve optimal performance, it is
vital to avoid unnecessary copies. Also, it is very important to not block on a single transfer, but to give an RNIC multiple
requests to work on and fill up its processing pipeline. 

This results in the interface shown in Listing~\ref{list:sender}. We send a message by providing a reference
to a previously allocated memory region, which asynchronously starts the message transfer. This asynchronous approach allows
us to start multiple concurrent transmissions. Throughout this thesis, we will talk about \emph{unacknowledged messages}, which
are messages that have been sent but their transfers have not yet been completed. These unacknowledged messages are very 
important to achieve the best possible performance.

\begin{figure}[htp]
\begin{lstlisting} [caption={Sender interface},captionpos=b, label={list:sender}] 
class Sender {
    // Takes a registered region containing the message
    // and initiates the transfer. Returns an ID 
    // associated with this request
    uint64_t SendAsync(SendRegion reg);

    // Blocks until the request with the given ID is
    // completed
    void Wait(uint64_t id);
}
\end{lstlisting}
\end{figure}

\paragraph{} To allow us to \code{Wait} for a specific work request, most of our implementations use the same process. 
Initiating a send request returns a monotonically increasing ID. This ID is also used as a \code{wr\_id} for the issued RDMA
operation. We store the last ID with a matching completion event. If we try to wait on a lower ID we know we already received 
a matching completion event prior, because of the in order guarantees of InfiniBand. If the provided ID is higher than the
last seen ID, we poll the completion queue until we receive one with a matching \code{wr\_id}.

This approach will eventually lead to an overflow, but given our observed  message rates we do not expect to see this 
happen in the next 10'000 years. 

\subsubsection{Receiver}

\paragraph{} Almost all of our implementations of a receiver have fixed regions of memory to receive messages into. 
For an application, receiving means that the protocol returns a reference to one of, or a part of, these memory regions. 

As soon as the application  has processed this message it needs to signal to the protocol that it can reuse the corresponding
buffer. We call this \emph{freeing} the receive buffer.

\begin{figure}[htp]
\begin{lstlisting} [caption={Receiver interface},captionpos=b, label={list:receiver}] 
class Receiver {
    // Waits for an incoming message and returns
    // the buffer containing it
    ReceiveRegion Receive();

    // Marks the previously received region 
    // as ready to be reused by the protocol
    void Free(ReceiveRegion reg);
}
\end{lstlisting}
\end{figure}







\pagebreak
\section{Send Receive} \label{sec:conn:send}\label{sendrcv}\label{sendrcv-design}

The send-receive protocol is by far the simplest, as the send and receive verbs already provide the required 
message passing interface. There are still multiple ways to implement and optimize such a protocol and a few pitfalls 
we need to address to get good performance.

\paragraph{} We presented the basic function of the send verb in Section \ref{sec:bg:send}. It allows the sender to transmit 
a message to the receiver, without any additional information, as long as the receiver has posted a receive buffer that is 
large enough to receive the message. To transform this verb to a fully functioning connection, we add two basic things. 
A kind of \emph{receive buffer management} that allows the receiver to receive multiple messages at the same time, 
and to get any reasonable performance we need to be able to send multiple messages asynchronously.

\subsection{Protocol} 

The sender assumes that the receiver is always ready to receive the message and has at least one receive buffer 
posted. This means sending essentially only involves issuing a send work request. As we already mentioned earlier, it is very 
important to keep the sending pipeline full to get good performance. Using \emph{unacknowledged messages} can increase
throughput by a factor of five or more.
We use the trick of issuing send requests with monotonically increasing \code{wr\_id}s, explained in Section~\ref{sec:protocols},
to be able to support multiple 
outstanding messages.

\paragraph{} The receiver always needs to have at least one posted receive buffer. We achieve this by having an array of 
posted receive buffers, which are initially registered during connection setup. When receiving a new message we return 
the corresponding buffer. We can match a completion event to the correct buffer by using the array index of the receive 
buffer as a \code{wr\_id} when posting the receive request. As soon as the application is done processing the message,
it marks it as free which will repost the corresponding buffer. By having multiple receive buffers it allows us to have 
numerous unacknowledged messages which improves performance drastically.

\paragraph{} It is important to note that in production systems there needs to be a way for the sender to notice whether 
enough receive buffers are ready. If there is no posted receive buffer available when a message is received,
the receiver  generates
a so-called \emph{Reader Not Ready} error. This will either cause a large back-off for the sender 
or even cause the connection
to break.

We observed this problem specifically for N:1 communication. This can be mitigated to some extend by optimizing receiving and
reposting buffers.




\subsection{Extensions}

There are multiple extensions to the described base protocol. The changes either improve performance, allow us to
share resources between connections, or enable different kinds of communication patterns.

\subsubsection{Inline Sending}
The send verb has a slight variation called \emph{inline send}. Inline sending means that instead of 
simply referencing the payload in a work request, the sending CPU directly copies it to the RNIC using MMIO. This prevents
the NIC from having to issue an addition DMA and can reduce latency for very small messages~\cite{anuj-guide}. It does,
however,
increase the load for the sending CPU. As we will see, the sending CPU is oftentimes the bottleneck so we did not further 
evaluate inline sending in this thesis. It can, however, be a viable optimization for small messages.


\subsubsection{Receive Batching} 
As described, there are significant penalties if the receiver is not able to keep up and stalls the sender.
This can be mitigated a little by optimizing the receiver. For one, instead of polling one receive CQ at a time we poll up to 32
at a time into an array of CQEs. We observed in microbenchmarks that this can improve the observed throughput by nearly half.
And by batching the reposting of receive buffers we can also further improve throughput.

\subsubsection{Shared Receive Queues} 
By having an array of $k$ posted receive buffers for each receiver, the memory reserved for receiving messages can grow quite
large. If a node has  open connections to $N$ other nodes, it needs to reserve  $N*k*max\_msg\_size$ bytes, even if 
the total request volume is quite small (i.e., we expect only burst of $k$ messages but from different nodes at different times).

We can reduce the total memory usage by using \emph{Shared Receive Queues~(SRQ)}. As the name already tells us SRQs allow us
to share receive queues between multiple QPs. This means, we can reuse a single receive queue
for multiple connections, allowing
multiple receivers to share the same receive buffers. This means the total memory usage does not grow with the number of
open connections, but stays constant.

The usage of SRQs does not impact completion queues. The completion event for consuming a posted receive buffer still ends up 
in the CQ of the corresponding QP.

One major change when using an SRQ is that we do not submit receive work requests to the respective QPs but to the single SRQ. This
means that we either have to introduce locking to access the SRQ, which introduces a significant performance penalty, or 
delegate the posting of the receive buffer to a single thread. Instead of reposting the buffer themselves, each connection 
enqueues the id of the to be posted buffer into a queue. The single reposting thread will dequeue and repost it. This also
allows us to have central batching for reposting the buffers. We opted to use the latter approach for our SRQ based 
implementation.

\subsubsection{Single Receiver} 
It is very common to have an N:1 communication pattern where a single server receives messages from multiple clients. This
could be achieved by simply round-robin over the $N$ connections. For this connection, however, we used the fact that we can 
associate a single completion queue with multiple queue pairs. This means, if we are in a \emph{single receiver mode}, all 
receive completion events will end up in a single CQ. Allowing us to poll a single queue to receive a message from $N$ 
different sender.


\subsection{Feature Analysis}

The Send and Receive verbs are generally regarded as the simplest verbs to work with and they give us a lot of features
that make them suitable for many applications. We take a look at the traits we introduced in Section~\ref{sec:features} 
and see which of them are met by our send-receive protocol.

\paragraph{} The protocol fulfills our requirements for being \emph{non-blocking}, which means if the receiving application does not free a single
received buffer, it does not impact the protocols ability to transmit new messages.
It also provides us with convenient \emph{interrupts} that allow us to save a lot CPU utilization for certain workloads.
Finally, when using shared receive queues, the protocol allows us to \emph{share resources} between multiple connections.

\paragraph{} It does, however, not allow for \emph{variable message sizes} which can unnecessarily blow up the total memory
requirements if an application needs to send many small and a few large messages, as each of the small messages will have
to consume a buffer that is large enough to contain a large message. We also argue that the protocol is not truly 
\emph{zero copy} as for almost all real world use cases we would need to copy the received message out of the
receive buffer and into a data structure.



\pagebreak
\section{Direct Write}\label{sec:conn:direct_write}

\subsection{Design} \label{sendrcv-design}
The idea of the \emph{Unbuffered} or \emph{Direct Write Protocol} is to avoid unnecessary copying we have when using
a buffered write connection with a ring buffer. We can achieve this by having the receiver specify specific memory 
regions for the sender to write to. 

\paragraph{} For this thesis we implemented something very reminiscent of the send receive protocol where the receiver 
posts receive buffer by sending the corresponding metadata to the sender, which will use these buffers in order. This does
not really make total use of this protocol as one would still need to copy the received message out of these buffers. This
could however fairly easily be extended by sending additional information with receive buffer, allowing the sender to write
the message to the correct buffer.\comment{not sure if this is clear / how to correctly explain this}


\begin{figure}[!htb]
\begin{center}
\begin{tikzpicture}[node distance=2cm,auto,>=stealth', minimum width=.75cm,minimum height=.5cm]
  \draw[rounded corners] (-.5, -3) rectangle (-4.5, 3) {};
  \node[align=center] at (-2.5, 2.5) {Sender};
  \draw[rounded corners] (.5, -3) rectangle (4.5, 3) {};
  \node[align=center] at (2.5, 2.5) {Receiver};

  \node [fill=white] at (-2.5, 1.5) (A) {\tiny addr};
  \node [anchor=west, fill=white] at (A.east) (A1) {\tiny rkey};
  \node [anchor=west, minimum width=.25cm, fill=white] at (A1.east) (A2) {\tiny v};

  \node [anchor=north, fill=black!35] at (A.south) (B) {\tiny a3};
  \node [anchor=west, fill=black!35] at (B.east) (B1) {\tiny 1};
  \node [anchor=west, minimum width=.25cm, fill=black!35] at (B1.east) (B2) {\tiny 0};

  \node [anchor=north, fill=black!35] at (B.south) (C) {\tiny b5};
  \node [anchor=west, fill=black!35] at (C.east) (C1) {\tiny 1};
  \node [anchor=west, minimum width=.25cm, fill=black!35] at (C1.east) (C2) {\tiny 0};

  \node [anchor=north, fill=black!05] at (C.south) (D) {\tiny c8};
  \node [anchor=west, fill=black!05] at (D.east) (D1) {\tiny 1};
  \node [anchor=west, minimum width=.25cm, fill=black!05] at (D1.east) (D2) {\tiny 1};
  \draw [<-,black!40!blue] (D.west) --  +(-.5,0)
        node [black!40!blue,left,inner xsep=.2, draw=none] (Tail) {\tiny next send};


  \node [anchor=north, fill=black!05] at (D.south) (E) {\tiny d2};
  \node [anchor=west, fill=black!05] at (E.east) (E1) {\tiny 1};
  \node [anchor=west, minimum width=.25cm, fill=black!05] at (E1.east) (E2) {\tiny 1};

  \node [anchor=north, fill=black!35] at (E.south) (F) {\tiny f3};
  \node [anchor=west, fill=black!35] at (F.east) (F1) {\tiny 1};
  \node [anchor=west, minimum width=.25cm, fill=black!35] at (F1.east) (F2) {\tiny 0};
  \draw [<-,black!40!green] (F2.east) --  +(.5,0)
        node [black!40!green,right,inner xsep=.2, draw=none] (head) {\tiny next free};



  \node [fill=black!35, minimum width=2cm] at (2.75, 1.5) (rB) {};
  \node [anchor=east, fill=white, minimum width=.5cm] at (rB.west) (rB1) {\tiny a3};
  \node [anchor=west, fill=black!35, minimum width=.5cm] at (rB.east) (rB2) {\tiny 1};

  \node [anchor=north, fill=black!35, minimum width=2cm, outer ysep=2] at (rB.south) (rC) {};
  \node [anchor=east, fill=white, minimum width=.5cm] at (rC.west) (rC1) {\tiny b5};
  \node [anchor=west, fill=black!35, minimum width=.5cm] at (rC.east) (rC2) {\tiny 1};

  \node [anchor=north, fill=black!05, minimum width=2cm, outer ysep=2] at (rC.south) (rD) {};
  \node [anchor=east, fill=white, minimum width=.5cm] at (rD.west) (rD1) {\tiny c8};
  \node [anchor=west, fill=black!05, minimum width=.5cm] at (rD.east) (rD2) {\tiny 0};

  \node [anchor=north, fill=black!05, minimum width=2cm, outer ysep=2] at (rD.south) (rE) {};
  \node [anchor=east, fill=white, minimum width=.5cm] at (rE.west) (rE1) {\tiny d2};
  \node [anchor=west, fill=black!05, minimum width=.5cm] at (rE.east) (rE2) {\tiny 0};

  \node [anchor=north, fill=black!35, minimum width=2cm, outer ysep=2] at (rE.south) (rF) {};
  \node [anchor=east, fill=white, minimum width=.5cm] at (rF.west) (rF1) {\tiny f3};
  \node [anchor=west, fill=black!35, minimum width=.5cm] at (rF.east) (rF2) {\tiny 1};

\end{tikzpicture}
\end{center}
\caption{Resources of the Direct Send Connection}
\label{fig:dirwrite-resources}
\end{figure}

Figure~\ref{fig:dirwrite-resources} is a representation of the data structures involved in the direct write 
connection. In the following subsections we will walk tough the steps involved in sending and receiving a message and
will reference these data structures.

\subsubsection{Sender}

The sender has an array of structs, which we will call \code{BufferInfo}, in local memory. This array is populated by the 
receiver and represent receive buffer to which the sender is allowed to write to. The struct contains three fields: 
\code{addr}, which is the virtual address of the corresponding buffer in the receivers memory, \code{rkey}, which is the 
rkey of that buffer, and \code{valid} which indicates whether this entry is actually valid.

\paragraph{} This array acts as a ring buffer, so when sending a message, the sender will look at the next field. If it is
valid it will send the message to the corresponding buffer at the receiver and invalidates it. We will always send the complete
buffer followed by a signaling byte, which will allow the receiver to poll on this byte to notice incoming messages.

\paragraph{} Our implementation of the sender is very simple, but it can easily be extended to be more sophisticated by adding
tags to the \code{BufferInfo} or by having multiple different arrays, which would allows us to \emph{route} the message to the
correct buffer at the sender and potentially eliminate the need for a further copy at the receiver. One could also change
the way of signaling an incoming message by switching to write with immediate or by applying a metadata approach as implemented
for the buffered write protocol.

\subsubsection{Receiver}

In a way the direct write receiver works in a similar way to the send and receive receiver. It can \code{Receive()} and 
\code{Free()} buffers. 

\paragraph{}\emph{Freeing} a buffer means making it available for the sender to write to. As we established at the sender 
there is a queue of \code{BufferInfo}. To \emph{free} a buffer we will first set the very last signal byte to 0, then we
write its address, rkey, and a valid byte of 1 to the next empty or invalidated queue entry. The receiver will keep a very 
similar local queue to keep track of the oder of posted buffers.

\paragraph{} To \emph{Receive} a buffer the receiver polls on the signal bit of the oldest freed buffer. This will be set
to one when the transmission is completed. We are again using the fact that RDMA updates memory in increasing memory order, 
for any modern systems~\cite{herd, farm}.


\subsection{Conclusion}

With our Direct Write connection we essentially rebuilt the Send and Receive verbs using only RDMA Writes. This gives us more
control over the protocol, which could allow us to extend it for specific systems.

\paragraph{} By adding metadata when posting a buffer this could enable \emph{True Zero Copy} capabilities for certain 
applications and with a more sophisticated buffer management one could more effectively utilizes the available memory by 
posting different size messages. The protocol also keeps the \emph{fairness} guarantees from the Send Receive connection. 
This gives us a very versatile protocol that can be adapted when necessary.

\paragraph{} But our benchmarking also showed that it is not trivial to achieve the same performance we got using the Send
and Receive verb. The general wisdom of RDMA writes being faster was not confirmed in our case. Our protocol can 
definitely be further optimized, by reposting buffers in batches or by redesigning this reposting entirely to reduce
the number of returning writes. But it showed that achieving good performance with such a protocol requires significant 
optimization work.









\pagebreak
\section{Buffered Write} \label{sec:conn:buf_write}

\subsection{Design}
As described in Section \ref{sec:proto-ds} there are many different ways to implement a buffer write protocol, or a
protocol that writes to a ring buffer. We decided that we want to allow for variable message sizes. That means 
in contrast to the send protocol presented in section \ref{sec:conn:send} we do not have a fixed maximum buffer size
that will always be consumed, but we will only use the space we actually need.

\paragraph{} As we discussed we have two distinct problems we need to solve to get a correct buffered write connection.

\begin{itemize}
  \item We need to signal to the receiver that we have written to the buffer and how much was written. This is tightly 
    coupled to \emph{how} we write the data. We implemented three different approaches for solving this, which we present
    in the \emph{Sending} section below.
  \item We need to signal to the sender whether there is enough space to write the message. We implemented both a \emph{push}
    and \emph{pull} based implementation in the \emph{Acknowledging} section below.
\end{itemize}


\subsubsection{Buffer} \label{sec:conn:write:buf}
\begin{figure}[!ht]
\begin{center}

\begin{tikzpicture}[>=latex,font=\sffamily,semithick,scale=2]
    \fill [black!35] (0,0) -- (130:1) arc [end angle=-15, start angle=130, radius=1] -- cycle;

    \fill [black!20] (0,0) -- (200:1) arc [end angle=160, start angle=200, radius=1] -- cycle;
    \draw [thick] (0,0) circle (1cm);

    \node (zero) at (90:1.1) {0};
    \draw[dashed] (90:1) -- (0:0);
    \draw (200:1) -- (0:0);
    \draw (175:1) -- (0:0);
    \draw (160:1) -- (0:0);

    \draw (130:1) -- (0:0);
    \draw (75:1) -- (0:0);
    \draw (55:1) -- (0:0);
    \draw (15:1) -- (0:0);
    \draw (-15:1) -- (0:0);
    \node [circle,thick,fill=white,draw=black,align=center,minimum size=3cm] at (0,0) {};


    \draw [<-,black!40!green] (130:1) -- (130:1.25) -- +(-.333,0)
        node [black!40!green,left,inner xsep=.333cm] (rptr) {Read Pointer};
    \draw [<-,black!40!green] (200:1) -- (200:1.25) -- +(-.333,0)
        node [black!40!green,left,inner xsep=.333cm] (Head) {Head};
    \draw [<-,black!40!blue] (-15:1) -- (-15:1.25) -- +(.333,0)
        node [black!40!blue,right,inner xsep=.333cm] (Tail) {Tail};
\end{tikzpicture}

\begin{tikzpicture}[>=latex,font=\sffamily,every node/.style={minimum width=1cm,minimum height=1cm,outer sep=0pt,draw=black,semithick, scale=0.5}]
        \node [minimum width=.5cm, fill=black!25] at (0,0) (A) {};
        \node [anchor=west, minimum width=.68cm, fill=black!35] at (A.east) (B) {};
        \node [anchor=west, minimum width=1.38cm, fill=black!35] at (B.east) (C) {};
        \node [anchor=west, minimum width=1.03cm, fill=black!35] at (C.east) (D) {};
        \node [anchor=west, minimum width=7.5cm] at (D.east) (E) {};

        \node [anchor=west, minimum width=1.38cm, fill=black!35] at (E.east) (F) {};

        \node [anchor=west, minimum width=.5cm, fill=black!10, draw=none] at (F.east) (A') {};
        \node [anchor=west, minimum width=.68cm, fill=black!10, draw=none] at (A'.east) (B') {};
        \node [anchor=west, minimum width=1.38cm, fill=black!10, draw=none] at (B'.east) (C') {};
        \node [anchor=west, minimum width=1.03cm, fill=black!10, draw=none] at (C'.east) (D') {};
        \node [anchor=west, minimum width=7.5cm, draw=none] at (D'.east) (E') {};
        \node [anchor=west, minimum width=1.38cm, fill=black!10, draw=none] at (E'.east) (F') {};
        \node [anchor=west, minimum width=12.47cm, draw=black!10] at (F.east) (A') {};

        \draw [<-,black!40!blue, scale=1] (1.7,.25) -- (1.7, .5)  -- +(.15,0)
        node [black!40!blue,right,inner xsep=.333cm, draw=none] (Tail) {\huge Tail};
        \draw [<-,black!40!green, scale=1] (5.42,-.25) -- +(0, -.25)  -- +(-.15,-.25)
        node [black!40!green,left,inner xsep=.333cm, draw=none] (Head) {\huge Read Pointer};

\end{tikzpicture}
\end{center}
\caption{Ring Buffer with twice mapped memory to allows DMA writes over the end}
\comment{Fix figure to include read pointer correctly}
\label{fig:ringbuffer}
\end{figure}

The core of our buffered read connection is a ring buffer. This buffer ensures that the sender always knows \emph{where} to 
write to. Figure \ref{fig:ringbuffer} shows the basis for our ring buffer that we use for all implementations.

\paragraph{} We use a buffer that allows us to write arbitrary message sizes. There are three relevant pointers. 
The \emph{tail} which points to the next free memory address, the \emph{head}, which points to  the end of the 
free memory region, and the \emph{read pointer} which points to the start of the next unprocessed message. 

\paragraph{} The tail is only stored and updated by the sender. The tail directly gives the sender the address to write the 
next message to. The only thing the sender needs to check is, if there is actually enough space in the buffer for the 
intended message. So it needs at least a semi regular update on the head position. We look into this problem in 
the \emph{Acknowledging} section below.

\paragraph{} Receiving is a little more involved. Similarly to how we implemented the receive buffer management in 
Section \ref{sec:conn:send} we want to have the notion of \emph{reading} which returns a pointer to the start of 
the next unread message and \emph{freeing} a message which allows us to reuse that section of the ring buffer. 

This is where the difference between the \emph{head} and the \emph{read pointer} becomes apparent. The read pointer points 
to the next message the receiver has not read, while the head points to the oldest message the receiver has not freed. Reading
and freeing messages should work while interleaving, so we need some kind of management of free buffer to update the head 
correctly.

We solve this in a very simple way by having a linked list containing the starting address of the received messages. When
reading we push the address to the end of the list. When freeing we remove it from the list. The head is always the first 
element in the list.

We should note that the length of the message has to be communicated in some way, which we will address in the section 
\emph{Sending} below.





\paragraph{"Magic" Buffer} One problem we encountered while using such a ring buffer for RDMA is that we cannot simply 
\emph{wrap around} the end of the buffer using DMA. So if we want to write a message that is larger then the space left until
the end of the buffer we either need to skip to the front of the buffer, wasting space, or perform two writes, which is 
significantly slower.

Figure \ref{fig:ringbuffer} shows the our problem. Message $A$ cannot be written in a single RDMA write as it consists of 
two distinct, not connected memory regions. We can solve this by using something that has been coined 
\emph{"Magic" ring buffer} \comment{By? rgiesen.wordpress.com? Someone else?}. The idea is to map the same physical twice,
so that the same physical memory is adjacent to itself in virtual memory. You can see a representation of this in the 
Figure above. We use \emph{shared memory segments} on Linux to achieve this memory layout.

With this change we can simply write over the end of the buffer and do not have to worry about wrapping around.

\subsubsection{Sending} \label{sec:conn:write:sender}
With the ring buffer described above, the sender always knows where to write the next message to. We still need to solve
the problem of notifying the receiver that we sent a message and communicate the size of this message.

\paragraph{}The sender and receiver interfaces stay exactly the same as presented in Section \ref{sec:conn:send}. We use the same 
monotonically increasing \code{wr\_id}s to wait for sends to complete. The freeing is explained in the buffer section
above. We implemented three different ways to send and receive


\paragraph{Write Immediate}

We introduced \emph{Write with Immediate} in Section \ref{sec:bg:write}, which basically allows us to implement our sending 
and receiving in a very similar way to the send connection presented in the last section.

\paragraph{} Write with Immediate sends a 32 bit value while also writing to the specified location. More importantly it 
consumes a receive buffer and creates a completion event at the receiver. This completion event also contains the number 
of bytes written. So by writing the message using write with immediate, we will generate a completion event at the receiver
which gives it the size of the message. The receiver can then immediately repost the received buffer and return the ring 
buffer segment. Through the in order guarantees of RDMA we know that the messages in the ring buffer will be in the same
order as the completion events in the queue. We can reuse both the receive buffer management and batching from the send 
connection.


\paragraph{Write Offset}

One key reason to use the write instead of send verb is that we do not need to generate a completion event at the receiver
and the common knowledge that writes are faster than sends~\cite{}\comment{Cite some papers that say this}. When we use 
write with immediate however, we expect the same performance as when using the send verb and we still need to handle receive
buffers and completion events. So we need other ways to notify the sender of incoming messages.

\paragraph{} One way to implement this is by having additional metadata that allows the receiver to notice incoming data.
We implement such a protocol with what we call \emph{Write Offset}. The idea is that together with each message, the sender
also updates a metadata region at the receiver containing the \emph{tail}. The receive can then notice new incoming messages
by polling this tail and comparing it to the \emph{Read Pointer}. We solve the problem of finding the size of the messages by
writing it in the first 4 bytes of the written segment.

\paragraph{} This means to send a message of size $s$ the sender prepends the size $s$ to the message and writes it to the
tail of the buffer. It then writes the updated tail to the metadata region at the sender. Of these two RDMA writes only the 
latter needs to be signaled and we can issue both of the work requests at the same time, mitigating the impact of having to 
perform two writes for a single message. The receiver will always poll its local copy of the tail. As soon as the tail does 
not equal it \emph{Read Pointer} it knows that there are outstanding messages. We can read the first 4 bytes to get the size
of the next message and can then read it and update the \emph{Read Pointer}.

\paragraph{} This connection has obvious drawbacks, as we need to issue two writes for a single message. But this way we 
circumvent the need of receive buffers and end up with a completely \emph{passive receiver}. We thought of other metadata
based implementations of buffered read. For example instead of pushing the tail update to the receiver with each write, the
receiver could actively pull the tail update using RDMA read when necessary. This could potentially outperform the push based
implementation in high bandwidth situations. We did however not implement and evaluate this pull based approach.


\paragraph{Write Reverse}

With our \emph{Write Reverse} connection implementation we are able to notify the receiver without an additional write or
consuming a receive buffer. We can do this by polling on the actual transmitted data. Previous work~\cite{herd, farm} has
shown that RDMA updates memory in increasing memory order, at least for any modern systems we know. This allows us to poll
on the highest memory address to check whether a transfer is complete.

\begin{figure}[!hb]
\begin{center}

\begin{tikzpicture}[>=latex,font=\sffamily,semithick,scale=2]
  \draw [draw=none, fill=black!20] (20:2) arc [end angle=95, start angle=20, radius=2] -- (95:1.75)  arc [end angle=20, start angle=95, radius=1.75] -- cycle;
  \draw [draw=none, fill=green!10] (95:2) arc [end angle=145, start angle=95, radius=2] -- (145:1.75)  arc [end angle=95, start angle=145, radius=1.75] -- cycle;
  \draw (160:2) arc [end angle=20, start angle=160, radius=2];
  \draw (160:1.75) arc [end angle=20, start angle=160, radius=1.75];

  \draw (145:2) -- (145:1.75);
  \draw (140:2) -- (140:1.75);
  \draw[decoration={
            text along path,
            text={0},
            text align={center},
            raise=0.15cm},decorate] (145:1.75) arc (145:140:1.75);
  \draw (100:2) -- (100:1.75);
  \draw[decoration={
            text along path,
            text={data},
            text align={center},
            raise=0.15cm},decorate] (140:1.75) arc (140:100:1.75);


  \draw (95:2)  -- (95:1.75);
  \draw[decoration={
            text along path,
            text={1},
            text align={center},
            raise=0.15cm},decorate] (100:1.75) arc (100:95:1.75);

  \draw [<-,black!40!blue] (95:2) -- (95:2.25) 
        node [black!40!blue,above,inner xsep=.333cm] (Tail) {Tail};

  \draw [<-,black!20!blue] (140:2) -- (140:2.25) 
        node [black!20!blue,above,inner xsep=.333cm] (ntail) {\small New Tail};



  \draw (60:2)  -- (60:1.75);
  \draw (55:2)  -- (55:1.75);

\end{tikzpicture}
\end{center}
\caption{Reversed Ring buffer}
\label{fig:write-rev}
\end{figure}

\paragraph{} So the core idea is to append a \emph{valid byte} at the end of the message on which the receiver can poll on.
There are two issues with introducing this two our existing ring buffer design.

\begin{itemize}
  \item The messages have variable size, so the receive does not know the location of this \emph{valid byte}.
  \item We need to zero this byte after freeing the buffer segment, potentially forcing us to zero the complete segment 
    which is fairly expensive.
\end{itemize}

We can solve both these problems by flipping our ring buffer. Instead of writing in increasing order, we add messages in 
decreasing order. Figure \ref{fig:write-rev} shows how writing a message works, the newly written message is highlighted in 
light green. We write a message structure of: 
A zero byte, followed by the data, the message length, and a valid byte, which is set to one. This has the effect that the
receiver can poll on the next byte in the ring buffer to check for new incoming messages and read the next 4 bytes to get 
the size of it. The prepended zero byte actually zeros the valid byte of the next massage. This means the receiver does not 
need to zero anything after freeing the segment.




We note that reversing the ring buffer does not really change any implementation details presented in 
Section~\ref{sec:conn:write:buf}. The double mapped memory trick still works and the freeing of buffers still works in a
similar way.

\subsubsection{Acknowledging}

We showed that there are multiple ways to transfer data and notify the sender. One other thing we need to prevent is for the
sender to overrun the ring buffer. That means the sender needs at leas a periodic update of the head position to decide if
there is space left to write to. We call this \emph{acknowledging} the head position.

We present two different approaches to do this. A pull based approach, where the sender reads the updated head from the
receiver, and a push based approach, where the receiver sends updates to the sender when the head position changes.

\paragraph{Pull} For the pull based implementation the receiver has a dedicated memory region where it writes the current 
head to. The sender has a cache of this head position. As soon as there is not enough space for the next message, given 
the cached head value, the sender will perform a RDMA read to update its cache.

Our current implementation will immediately block until the read succeeds. There is potential to optimize this by preemptively 
updating the head without waiting for it to complete. We experimented adding this, but as we did not see any significant 
performance improvements by adding this, we decided not to add this to reduce complexity. \comment{Maybe we should add it. Its ready}

\paragraph{Push} We also implemented a push based approach, where the receiver will send updates of the tail value using 
RDMA sends. To reduce the load of this acknowledgements we only send updates when the receiver has processed an eighth of 
the complete ring buffer. The sender will then poll its receive queue as soon as it cannot write the next message given its
cache of the tail value.

\subsection{Evaluation}

We ran all our evaluations on two machines running CentOS 7 containing two Intel Xeon Gold 6152 and 384 GiB of memory.
The two nodes each contain a Mellanox ConnectX-5 (100Gbps) and are connected through a 100 Gbps switch. All measurements
have been performed using RoCE.

\subsubsection{Latency}

\begin{figure}[h]
\includegraphics[width=1\textwidth]{write-lat-msgsize.png}
\caption{Write latency between two nodes}
\label{fig:plot-write-lat}
\end{figure}

Figure \ref{fig:plot-write-lat} shows the latency of sending a message of varying sizes using our write protocol
with all presented sender. All measurements use the \emph{read} acknowledger. There is no significant difference
in latency when switching to a \emph{send} acknowledger. \comment{do we need a graph for that?}
We again perform \emph{ping-pong} measurements. We take half of this RTT as our measurement of latency.


We can see that the additionally issued write of the \emph{WriteOff} gives us a noticeable increase in latency
for all message sizes. The overhead however does seem to decrease a little for messages over the MTU of $4096$
Bytes. This is something we are not quite able to predict with our model.

The latency characteristics of \emph{WriteRev} and \emph{WriteImm} seem to be very similar. This is in line 
with what we expect. The consumption of a receive request doe not seem to introduce a significant overhead.
\comment{the slight slowdown of writeRev might be related to our anomaly}

\subsubsection{Bandwidth}

\begin{figure}[h]
\includegraphics[width=1\textwidth]{write-bw-msgsize.png}
\caption{Write bandwidth with 64 unacknowledged messages and send acknowledging}
\label{fig:plot-write-bw}
\end{figure}

Figure~\ref{fig:plot-write-bw} shows the measured bandwidth for our implementation with the different presented
senders. We only show the measurements using the \emph{send} acknowledger as, with one exception we will 
explain below, there are not significant differences in performance between these two implementations.

\paragraph{} We ran all measurements with 64 unacknowledged messages, as this was enough to saturate our 
devices pipeline in all cases. We did not implement sender side batching as presented in 
Section~\ref{sec:conn:send}. We would expect to see similar improvements for smaller message sizes, but 
decided not to focus on that as whenever possible application level batching is more effective.

\paragraph{} We can see that both \emph{Write Reverse} and \emph{Write Immediate} sender achieve very similar
high performance. The \emph{Write Offset} sender shows consistently lower throughput until it also achieves
link speed for message sizes of 16 KB. This is what we would expect as this implementation needs to issue 
twice as many operations. It seems that we are bottleneck by the sending NIC overhead when the other
implementations already achieve line rate. We also see that the generation of completion events at the sender
does not seem to be a bottleneck.

\begin{figure}[h]
\includegraphics[width=1\textwidth]{write-bw-rev-anom.png}
\caption{Write Reverse bandwidth for message sizes around 4 KB with read acknowledgements}
\label{fig:plot-write-rev-anom}
\end{figure}

\paragraph{Write Reverse Anomaly} During our evaluation we encountered a anomaly we cannot fully explain at
this point. A seen in Figure  When using a \emph{write reverse} sender and a \emph{read} acknowledger we see 
significant drop in performance when sending messages that are slightly larger than 4090 bytes. It is worth 
noting that when adding the 6 byte overhead from the protocol explained in Section~\ref{sec:conn:write:sender}
this happen exactly when we the final write is larger than 4 KB, which is both the pagesize and MTU. Bandwidth
then seems to linearly increase and will drop again very similarly for multiples of 4 KB.

\paragraph{} Interestingly this cannot be observed with other sender implementation and the effect is greatly
reduce when using the send acknowledgements. We suspect this could be caused by our unusual write direction 
which causes suboptimal NIC cache usage and results in may cache misses while writing.\comment{Would like 
to have input on that. I'm not sure what is happening and how to explain this}
But this is just one  possible explanation and further research would be necessary to fully explain these
results.

\paragraph{} For all other plots we purposely avoid to exactly hit this window for the \emph{write reverse}
sender but will send slightly smaller messages.

\subsubsection{Multithreading}


\begin{figure}[]
  \centering
\begin{subfigure}[b]{0.49\textwidth}
  \centering
  \includegraphics[width=1\textwidth]{write-bw-threads-16.png}
  \caption{Message size 16 bytes}
  \label{fig:plot-write-bw-thread-16}
\end{subfigure}
\begin{subfigure}[b]{0.49\textwidth}
  \centering
  \includegraphics[width=1\textwidth]{write-bw-threads-512.png}
  \caption{Message size 512 bytes}
  \label{fig:plot-write-bw-thread-512}
\end{subfigure}
\begin{subfigure}[b]{0.49\textwidth}
  \centering
  \includegraphics[width=1\textwidth]{write-bw-threads-8192.png}
  \caption{Message size 8192 bytes}
  \label{fig:plot-write-bw-thread-8192}
\end{subfigure}
  \caption{Bandwidth with varying number of threads}
\end{figure}
\subsubsection{Single Receiver}

\begin{figure}[]
  \centering
\begin{subfigure}[b]{0.49\textwidth}
  \centering
  \includegraphics[width=1\textwidth]{send-bw-unack.png}
  \caption{Bandwidth with message size of 16 bytes with varying number of unacknowledged messages}
  \label{fig:plot-sndrcv-bw-unack}
\end{subfigure}
\begin{subfigure}[b]{0.49\textwidth}
  \centering
  \includegraphics[width=1\textwidth]{send-bw-batch.png}
  \caption{Bandwidth with message size of 16 bytes with varying batch size}
  \label{fig:plot-sndrcv-bw-batch}
\end{subfigure}
\begin{subfigure}[b]{0.49\textwidth}
  \centering
  \includegraphics[width=1\textwidth]{send-bw-msgsize.png}
  \caption{Evaluation of the Send Receive bandwidth between two nodes and our performance model}
  \label{fig:plot-sndrcv-bw}
\end{subfigure}
\end{figure}

\subsection{Conclusion}










\pagebreak
\section{Shared Write}
\subsection{Protocol}
\subsection{Device Memory}
\subsection{Evaluation}
\todo{Info on system etc.}

\subsubsection{Latency}

\begin{figure}[h]
\includegraphics[width=1\textwidth]{write-atomic-lat-msgsize.png}
\caption{Evaluation of the Shared Write latency between two nodes}
\label{fig:plot-write-atomic-lat}
\end{figure}

Figure \ref{fig:plot-write-atomic-lat} shows the latency of our \emph{Shared Write} protocol between two nodes. 
We again show half the ping-pong round-trip time. We can clearly see that we have a large constant overhead, caused
by the reserving phase of over $2L$ independent of the message size.

We are reduce this constant overhead by about 0.25 $\mu$s by using device memory for the connections metadata. This is 
in line with what we expect given our micro benchmark, and is caused by the receiving NIC not having to access the 
receivers RAM.


\subsubsection{Bandwidth}

\begin{figure}[h]
\includegraphics[width=1\textwidth]{write-atomic-bw-msgsize.png}
\caption{Evaluation of the Shared Write bandwidth between two nodes}
\label{fig:plot-write-atomic-bw}
\end{figure}

In figure \ref{fig:plot-write-atomic-bw} we can see the throughput of the \emph{Shared Write} protocol for a single 
connections, with varying message size. 

\todo{This is very slow, we would need to redesign the connection a little, as right now we can only support 2 unack 
messages}

show linear


\subsubsection{Resource Sharing}

\begin{figure}[h]
\includegraphics[width=1\textwidth]{write-atomic-bw-threads.png}
\caption{Shared Write bandwidth with a single receiver and varying number of sender}
\label{fig:plot-write-bw-unack}
\end{figure}








\pagebreak

\section{Direct Read} \label{sec:conn:direct_read}
\subsection{Design}

In Section~\ref{sec:conn:direct_write} we discussed how we can possibly avoid an additional copy at the receiver by giving 
the sender information which allows him to potentially write the data to the correct final memory location. The next logical
step is to let the receiver decide for each message where to write it to. We can achieve this by our implementation of a
\emph{Direct Read Connection}.

\paragraph{} The core idea of a direct read protocol is that instead of directly sending a message through a send or write 
verb, the sender simply notifies the receiver of a new message and where it is located, we will call this a \emph{Read Request}.
The receiver then issues an RDMA read operation to directly read this message to the correct location.

By allowing to sender to attach additional domain specific information to this read request, the receiver can directly 
move this data to the correct space in a data structure, or potentially even directly NVMe storage for certain applications.
\comment{I remember correctly this is possible right? I had a hard time finding a relevant paper}


\subsubsection{Sender}
The sender interface is again largely unchanged and consists of a \code{SendAsync()} method which takes a buffer containing the
message and a \code{Wait()} method that waits until the transfer is completed.

\paragraph{} Instead of sending the complete buffer, the sender only sends a small \emph{Read Request} using an inline send,
containing the location of the buffer. It is then the job of the receiver to access this buffer.

\paragraph{} To wait for the transfer to be completed, and for the buffer to be able to be reused, we can not simply wait 
for the completion event of the send, like we do for the send or write based connections. We need to wait for the receiver 
to explicitly signal that the buffer was transfered. We append a signaling byte at the end of the sending buffer. 
When sending this byte will be set to 0 and we can wait for the transport to be completed by polling this byte until the 
receiver will update it.

This push based implementation introduces little additional complexity, but there are many different ways to implement such 
signaling. The signaling bit forces us to use a specific memory arrangement, which could prevent us to send data directly 
from certain data structures. In such cases a pull based approach or an implementation using send might be a better approach
and depending on the implementation could actually result in better performance.

\subsubsection{Receiver}
The receiver polls the receive queue and receive the \emph{Read Request}. It then instantly reposts the associated 
receive buffer. With this information it then issues a read request for the message. As for this connection the receiver is
doing the heavy lifting, it is crucial that we do not block until the read is completed to get reasonable performance. 

\paragraph{} This means the receiver has a slightly different interface that the previously presented connections. 
We split the receive call into a \code{RequestAsync} and a \code{Wait} method. The \code{RequestAsync} takes a receive buffer
to read into. It will wait for an incoming read request and issue the corresponding read. It uses the same increasing 
\code{wr\_id} approach we use for sending with which the \code{Wait} method can wait for the read to complete. This approach
allows us to pipeline receives the same way we pipeline sends.

\paragraph{} To update the signal byte to notify the that the transfer is complete we implement to different approaches.

Our first approach is to issue the write with the update at the same time as we issue the read. This however introduces a 
problem. While RDMA guarantees that two consecutive will happen in the issued order, it will not guarantee us that reads issued
before writes will be completed before we can observe the write~\cite{}. \comment{I think we need some kind of references for
that} To still be able to issue these two operations at the same time, we need to use a \code{IBV\_SEND\_FENCE}. When we add 
fence indicator to a write request its processing will not begin until all previous read and atomic operations on the same QP 
have completed. This allows us to issue both the read as well as the acknowledging write at the same time.

The other approach is simply to wait for the read to be completed before issuing the acknowledging write. In other words the
write will be posted as soon as \code{Wait} returns. That way we can avoid the fence which can be quite expensive.

\subsection{Evaluation}

\subsubsection{Latency}

Figure~\ref{fig:plot-dirread-lat} shows the point to point latency of the Direct Read connection both with and without
fencing. We can make two key takeaways.

\paragraph{}For one the latency is generally very high and is not as strongly affected by the message size, it remains more or
less constant up to a message size of 1 KB. This is caused by the large, constant overhead of sending a \emph{Read Request},
which adds half a round-trip time to each request, and the property of the RDMA read verb that its completion time remains 
more or less constant up to message size of \todo{1024 bytes}.

The other takeaway is that the fenced version is nearly $3 \mu s$ slower then the implementation without fencing. This is to 
be expected as the fenced version has the acknowledging of the completed in its critical path, while the fence-less 
implementation does not.\comment{Is this understandable?} So while fencing reduces the complete request time for the sender
it increases the actual latency significantly.


\begin{figure}[htp]
  \centering
\begin{subfigure}[b]{0.49\textwidth}
  \includegraphics[width=1\textwidth]{dir-read-lat-msgsize.png}
  \caption{Latency}
  \label{fig:plot-dirread-lat}
\end{subfigure}
\begin{subfigure}[b]{0.49\textwidth}
  \centering
\includegraphics[width=1\textwidth]{dir-read-bw-msgsize.png}
  \caption{Bandwidth}
  \label{fig:plot-dirread-bw}
\end{subfigure}
\caption{Measurements between two nodes with varying message size}
\end{figure}


\subsubsection{Bandwidth}

Figure~\ref{fig:plot-dirread-bw} shows the point to point throughput of the Direct Read connection using a single connection.
We allow for 32 unacknowledged messages for both the sender and receiver. We did not see any performance improvements above
this.

\paragraph{} The main thing we notice is that the fenced version achieves drastically lower throughput. The fence essentially 
serializes the reads and prevents the NIC to effectively pipeline operations. This gives us the on a low level linearly 
increasing bandwidth we observe.

\paragraph{} While the direct read actually works very differently from what we represent in our model the performance 
characteristics of the unfenced Direct Read connection are very similar to what we observed for previous connections. The
throughput increases linearly for small messages as we are limited by the number of request we are able to post. For larger
messages we are limited by the NIC giving us this sub-linear curve we see for almost all connections.

\paragraph{} For all further measurement we only focus on the unfenced implementation as the fenced version is generally unable
to achieve comparable throughput.

\subsubsection{Multithreading}
Figure~\ref{fig:plot-dirread-bw-threads} shows the protocols performance with multiple open connections. We still work with 
32 unacknowledged messages for both the sender and receiver and the throughput is again evaluated for three different message 
sizes: 16 bytes, 512 bytes, and 8192 bytes. 

\paragraph{} The results look very similar to other connections. There is a linear increase in performance, both by being 
able to post more work request for smaller messages and being able to utilize more NIC processing units.~\cite{anuj-guide}

When using 8192 byte messages, we are limited by the maximum link speed. When using 512 byte messages we hit a bottleneck at
around 70 Gbit/s, which is in line with what we observe with other protocols that need to issue two operations per message, 
like the Direct Write or Write Offset connection.



\begin{figure}[ht]
  \begin{subfigure}[b]{0.49\textwidth}
  \centering
  \includegraphics[width=1\textwidth]{dir-read-bw-threads.png}
  \caption{Using $N$ receivers}
  \label{fig:plot-dirread-bw-threads}
  \end{subfigure}
  \begin{subfigure}[b]{0.49\textwidth}
  \centering
  \includegraphics[width=1\textwidth]{dir-read-bw-n1.png}
  \caption{Using a single receiver}
  \label{fig:plot-dirread-bw-n1}
  \end{subfigure}
  \caption{Bandwidth with varying number of connections $N$}
\end{figure}

\subsubsection{Single Receiver}

Figure~\ref{fig:plot-dirread-bw-n1} shows the protocols throughput with a single receiver and varying number of senders.
We again look at three different message sizes. 

\paragraph{} Even more so than for previous protocols we are very much limited by the receiving CPU in this case, as the 
receiver already needs to do the heavy lifting for this protocol. 

For smaller messages there is no throughput improvements at all with increasing number of connections as we already have 
been limited by the ability of the receiving CPU to post work requests. For very large messages there is a small improvement 
by the ability to use multiple NIC processing units, but these are then quickly limited by the maximum link speed.

\paragraph{} This communication pattern does not seem to be good fit for this protocol as it is always limited at around 
2.6 MOp/s by the receiving CPU speed. This gives us the 10 Gbit/s and 0.3 Gbit/s bottleneck for the 512 byte and 16 byte size
messages respectively.

\subsection{Conclusion}

The Direct Read connection is very different approach to a data transfer protocol with the goal of achieving 
\emph{True Zero Copy} capabilities as the receive itself can read the message to its final place, preventing any additional 
copying. The protocol also provides us with good \emph{Fairness} guarantees and efficient memory usage though 
\emph{Variable Message Sizes} and not having any fixed buffers. With that \emph{Resource Sharing} follows as multiple
connections can reuse the same receive and send buffers. It also allows us to use \emph{Interrupts} form completion events
giving us low CPU utilization in low load situations as no component needs to constantly poll.

\paragraph{} All these features make this protocol viable, especially when handling large messages. However the very high
base latency and the impact of the communication overhead on bandwidth makes this not a good fit for transferring small 
messages. Further as the bulk of the work has to be done by the receiver this protocol is not ideal for N:1 communication
either.

\paragraph{} We saw that while \emph{fences} can be very powerful to ensure correctness, the can introduce severe performance
penalties and oftentimes simply waiting for the first operation to complete yields better throughput. So fences need to be
applied with care.

\paragraph{} The Direct Read protocol is not a good fit for all situations but can be a very good choice when transferring 
large amounts of data for for 1:N communication, for example when replicating data to other nodes. 
\comment{Actually saw a paper on distributed FS that uses this. Should we add such references to this section or simply leave
it in related work?}



\pagebreak

\section{Buffered Read}\label{sec:conn:buf_read}
\subsection{Design}

\begin{figure}[!ht]
\begin{center}
\begin{tikzpicture}[node distance=2cm,auto,>=stealth']
  \seqnode{B_cpu}{RAM};
  \seqnode[left of=B_cpu]{B_nic}{NIC};
  \hseqnode[right of=B_cpu, node distance=1.5cm]{B_acpu}{};
  \seqnode[left of=B_cpu, node distance=7cm]{A_cpu}{CPU / RAM};
  \seqnode[right of=A_cpu]{A_nic}{NIC};
  %
  \msg{B_acpu}{B_nic}{.15}{read WR MMIO}
  \msg{B_nic}{A_nic}{.2}{transfer read request}
  \msg{A_cpu}{A_nic}{.3}{meta DMA}
  \msg{A_nic}{B_nic}{.35}{transfer meta}
  \msg{B_nic}{B_cpu}{.4}{meta DMA}
  \msg[below]{B_nic}{B_cpu}{.45}{WQE DMA}
  \fetch{B_acpu}{B_cpu}{.5}{poll CQ}

  \msg{B_acpu}{B_nic}{.6}{read WR MMIO}
  \msg{B_nic}{A_nic}{.65}{transfer read request}
  \msg{A_cpu}{A_nic}{.75}{payload DMA}
  \msg{A_nic}{B_nic}{.8}{transfer payload}
  \msg{B_nic}{B_cpu}{.9}{payload DMA}
  \msg[below]{B_nic}{B_cpu}{.95}{WQE DMA}
  \fetch{B_acpu}{B_cpu}{1}{poll CQ}

  \end{tikzpicture}
\end{center}
\caption{Buffered Read sequence}
\label{fig:seq-buf-read}
\end{figure}

\todo{This is actually worse when there is no message ready. If the sender updates the head just at the most inoppertune moment
we would waste a whole meta read request}

\subsection{Evaluation}



\pagebreak
\section{Performance Model}\label{sec:perf-model} \label{sec:model}
In this section we introduce a general performance model for data exchange protocol that will allow us to estimate the 
performance of our protocol implementations. It gives us a better understanding of our evaluation results and helps us 
to locate bottlenecks.

\paragraph{}We took a fairly detailed look at the operations involved in transmitting a message using RDMA in Section~\ref{sec:rdma}.
Trying to closely model this however gives us a far too complex model, with a lot of parameters that are hard to assess,
especially if we start to look at more complex protocols. We introduce a more simplified model inspired by the  
\emph{LogGP model}~\cite{loggp}, which was developed  to model point to point communication with variable message sizes for 
traditional IP based networks.

\paragraph{} At its core the LogGP model uses a fixed CPU overhead $o$ per message, the communication latency $L$, and 
the bandwidth $G$. While this results in a decent model for IP based system, it is unable to model the heavy NIC offloading 
happening in RDMA. We extended the LogGP model by splitting the offset $o$ into multiple offsets for each of the components.
We also largely ignored the number of processes $P$ and use a slightly different definition for the gap $g$. This leaves 
us with the following parameters, which are illustrated in Figure~\ref{fig:model-base}.

\begin{figure}[!htp]
\begin{center}
\begin{tikzpicture}[node distance=1cm,auto,>=stealth']
  \node[] (p0) {P0};
  \node[right of=p0, node distance=10cm] (p0_g) {};
  \draw[dotted] (p0) -- (p0_g);

  \node[below of=p0, node distance=0.5cm] (p0_nic) {P0 NIC};
  \node[right of=p0_nic, node distance=10cm] (p0_nic_g) {};
  \draw[dotted] (p0_nic) -- (p0_nic_g);

  \node[below of=p0_nic, node distance=1.5cm] (p1_nic) {P1 NIC};
  \node[right of=p1_nic, node distance=10cm] (p1_nic_g) {};
  \draw[dotted] (p1_nic) -- (p1_nic_g);

  \node[below of=p1_nic, node distance=0.5cm] (p1) {P1};
  \node[right of=p1, node distance=10cm] (p1_g) {};
  \draw[dotted] (p1) -- (p1_g);
  

  %%%%

  \draw[very thick] (p0) --node[above,scale=0.75,midway]{$o_{snd}$} ($(p0)!0.15!(p0_g)$);
  \draw[very thick] ($(p0_nic)!0.15!(p0_nic_g)$) --node[above,scale=0.75,midway]{$o_{nsnd}$} ($(p0_nic)!0.225!(p0_nic_g)$);


  \path[] ($(p0_nic)!0.225!(p0_nic_g)$) --node[above,scale=0.75,midway]{$G$} ($(p0_nic)!0.26!(p0_nic_g)$);
  \draw[dotted, ->] ($(p0_nic)!0.225!(p0_nic_g)$) -- ($(p1_nic)!0.395!(p1_nic_g)$);

  \path[] ($(p0_nic)!0.26!(p0_nic_g)$) --node[above,scale=0.75,midway]{$G$} ($(p0_nic)!0.295!(p0_nic_g)$);
  \draw[dotted, ->] ($(p0_nic)!0.26!(p0_nic_g)$) -- ($(p1_nic)!0.43!(p1_nic_g)$);

  \path[] ($(p0_nic)!0.295!(p0_nic_g)$) --node[above,scale=0.75,midway]{$G$} ($(p0_nic)!0.33!(p0_nic_g)$);
  \draw[dotted, ->] ($(p0_nic)!0.295!(p0_nic_g)$) -- ($(p1_nic)!0.465!(p1_nic_g)$);

  \draw[dotted, ->] ($(p0_nic)!0.33!(p0_nic_g)$) -- ($(p1_nic)!0.5!(p1_nic_g)$);


  \draw[very thick] ($(p1_nic)!0.5!(p1_nic_g)$) --node[above,scale=0.75,midway]{$o_{nrcv}$} ($(p1_nic)!0.55!(p1_nic_g)$);
  \draw[very thick] ($(p1)!0.30!(p1_g)$) --node[above,scale=0.75,midway]{$o_{post}$} ($(p1)!0.345!(p1_g)$);
  \draw[very thick] ($(p1)!0.55!(p1_g)$) --node[above,scale=0.75,midway]{$o_{rcv}$} ($(p1)!0.6!(p1_g)$);
    
  %%%%

  \draw[dotted] ($(p0_nic)!0.225!(p0_nic_g)$) -- ($(p0_nic)!0.225!(p0_nic_g)+(0,-3)$);
  \draw[dotted] ($(p0_nic)!0.330!(p0_nic_g)$) -- ($(p0_nic)!0.330!(p0_nic_g)+(0,-3)$);
  \draw[<->] ($(p0_nic)!0.225!(p0_nic_g)+(0,-2.8)$) --node[above,scale=0.75,midway]{$(k-1)G$} ($(p0_nic)!0.330!(p0_nic_g)+(0,-2.8)$);

  \draw[dotted] ($(p1_nic)!0.5!(p1_nic_g)$) -- ($(p1_nic)!0.5!(p1_nic_g)+(0,-1.5)$);
  \draw[<->] ($(p0_nic)!0.330!(p0_nic_g)+(0,-2.8)$) --node[above,scale=0.75,midway]{$L$} ($(p1_nic)!0.5!(p1_nic_g)+(0,-1.3)$);


  \draw[dotted] ($(p0)!0.15!(p0_g)$) -- ($(p0)!0.15!(p0_g)+(0,.8)$);
  \draw[dotted] ($(p0)!0.65!(p0_g)$) -- ($(p0)!0.65!(p0_g)+(0,.8)$);
  \draw[<->] ($(p0)!0.15!(p0_g)+(0,.6)$) --node[above,scale=0.75,midway]{$g$} ($(p0)!0.65!(p0_g)+(0,.6)$);

  %%%%

  \draw[dotted, ->, lightgray] ($(p1_nic)!0.56!(p1_nic_g)$) -- ($(p0)!0.65!(p0_g)$);


\end{tikzpicture}
\end{center}
\caption{Sending an receiving messages under our model}
\label{fig:model-base}
\end{figure}




\begin{itemize}
  \item $L$: the network Latency, incurred by sending a message from the senders NIC to the receivers NIC. In our model 
    this also includes the PCI latency.
  \item $G$: the gap per byte for long messages. For our purposes this is the time of sending a single byte given the 
    maximum bandwidth of our link.
  \item $o_{snd}$: the \emph{send overhead}, defined as the length of time that a processor is engaged in sending each message.
    For some protocols this also includes any preparation and communication overhead necessary to send a message.
  \item $g$: the \emph{send gap}, defined as the minimum time interval until the sender can reuse the resources involved in 
    the transmission. (e.g. the send buffer) 
  \item $o_{nsnd}$: the \emph{send NIC overhead}, defined as the length of time that a NIC is engaged in sending each message.
  \item $o_{nrcv}$: the \emph{receive NIC overhead}, defined as the length of time that a NIC is engaged in receiving each message.
  \item $o_{rcv}$: the \emph{receive overhead}, defined as the length of time that a processor is engaged in receiving each message.
  \item $o_{post}$: the \emph{posting overhead}, defined as the time that a processor is engaged in preparing a receive buffer
    to receive into (e.g. post receive). This needs to happen sometime before receiving the next message.
\end{itemize}

\paragraph{} This model does not take the MTU into account. While some prior work use a slightly different model when sending 
messages over the MTU, that span multiple transmission units~\cite{dare}, we found our model to works well enough to understand
most of our results.

\paragraph{} We built this model around send or write based protocols and it was not designed 
for read based protocols. But as we will see it can still give us better insight in the observed performance of these 
protocols.

\paragraph{} Even with these simplifications it still is a fairly complex model and measuring each of these overheads in 
practice is very hard. We do not try to quantitatively evaluate our model for all presented protocols, but rather use it 
get a better general understanding of the observed performance characteristics and to understand what is limiting throughput
or latency in which situations.

\pagebreak
\subsection{Evaluating the Model}

\paragraph{} In this section we evaluate the model for the send-receive protocol only. Table~\ref{tab:model} shows our model
parameter estimates. 

\begin{table}[!ht]
\setlength\tabcolsep{1.5pt}
\centering
\caption{Model parameter for send-receive protocol}
\label{tab:model}
\begin{threeparttable}[t]
 \begin{tabular}{|x{3.5cm}|x{3.5cm}|x{3.5cm}|} % I specify the sizes of columns. x is for centering and p is  for left
 \hline
 Parameter    &  single & batched \\
  \hline
  \hline
 $o_{snd}$    & $0.12 \mu s$ &  $0.016 \mu s$\\
  \hline
 $o_{rcv}$    & $0.052 \mu s$ & $0.017 \mu s$\\
  \hline
 $o_{post}$   & $0.016 \mu s$ & $0.007 \mu s$\\
  \hline
 $o_{nsnd}$   & $0.02 \mu s$\tnote{a} &  \cellcolor{black!40} \\
  \hline
 $o_{nrcv}$   & $0.02 \mu s$\tnote{a} &  \cellcolor{black!40} \\
  \hline
 $L$          & $1.9 \mu s$ &  \cellcolor{black!40}\\
  \hline
 $g$          & $\geq 3.4 \mu s$\tnote{b}  & \cellcolor{black!40}\\
 \hline
 $G$          & $0.2\mu s / KB$  & $0.084\mu s / KB$\\

\hline
\end{tabular}
\begin{tablenotes}
\item[a] \small The on NIC overheads are expected to be very small and are hard to accurately measure
\item[b] \small Very protocol dependent. Increases linearly for the send-receive protocol
\end{tablenotes}
\end{threeparttable}
\end{table}


\paragraph{} We are able to directly measure $o_{snd}$, $o_{rcv}$, and $o_{post}$. For the send-receive protocol $o_{snd}$ is the cost of 
posting a single send request. The receiving overhead $o_{rcv}$ is the cost of polling for a receive completion event. And
$o_{post}$ is the cost for posting a single receive request. For all of these overheads we measure both the cost of a single
operation, as well as the cost per message when issuing multiple operations in a batch.

\paragraph{} The gap $g$ is also directly measurable. This is the time it takes until the send is completed and has been
acknowledged. The lower bound for this is at $3.4\mu s$, or a bit less then twice the network latency L. But it increases
linearly with the size of the message $k$.

It is worth noting here, that our reported network latency $L$ includes both the PCI latency as well as the latency
of the switch between our two nodes.

\paragraph{} For the network bandwidth G we also report a batched and unbatched estimation. We use the unbatched estimation
for our latency prediction and the batched estimation for our throughput predictions.

\paragraph{} Evaluating the RNIC overhead is tricky. We make the assumption that $o_{nsnd} = o_{nrcv}$ and use our latency
and bandwidth measurement to estimate these parameters. This gives us a good enough estimation for a 1:1 communication pattern.




\paragraph{} While we will see that our model is not quite able to accurately predict our bandwidth and latency 
measurements, it does  give us a better understanding of our results and predicts the different types of bottlenecks 
we encounter in our evaluation.

\subsubsection{Predicting Latency}

Predicting latency using our model essentially means evaluating the parameters in the critical path and adding them up. 
The predicted latency $t$ of transferring a single message of size $k$ is:

$$
t \geq o_{snd} + o_{nsnd}  + (k-1)G + L + o_{nrcv} + o_{rcv}
$$


\begin{figure}[ht]
  \centering
  \includegraphics[width=1\textwidth]{send-lat-msgsize.png}
  \caption{Send-Receive latency and model prediction}
    \label{fig:model-lat}
\end{figure}

\paragraph{} Figure~\ref{fig:model-lat} shows our prediction and the actual latency measurements for the send-receive protocol.
We can see that we are not quite able to precisely match the actual data. It does
however confirm our prediction of a mostly linear increase in latency with increased message size.


\subsubsection{Predicting Bandwidth}
Predicting maximum bandwidth is not quite as simple as adding up all parameters. The model predicts three different kinds of
bottlenecks. We take a look at each of these bottlenecks and illustrate them on the send-receive protocol in Figure~\ref{fig:model-bw}.

\begin{figure}[ht]
  \centering
  \includegraphics[width=1\textwidth]{send-bw-msgsize.png}
  \caption{Send-Receive bandwidth and model prediction}
    \label{fig:model-bw}
\end{figure}

\paragraph{Round-Trip bottleneck} One way to be bottlenecked is by simply not issuing enough concurrent send requests. In 
this case we are bottlenecked by the gap $g$, which usually contains a complete round-trip time and grows linearly with 
the size of the message. So when we issue only a single send at a time, for a message size of $k$ we predict to be limited by

$$
bw \leq \frac{k}{o_{snd} + g}\text{, where } g = g_{fix} + kg_{var} 
$$

For the send-receive protocol this is exactly what we see. SR-Seq shows the throughput with varying message size when only 
sending a single message at a time. 

\paragraph{} This means to get decent 1:1 performance it is vitally important to issue enough concurrent requests, or as we 
will call it from now on to allow for enough \emph{unacknowledged} messages. In a lot of related work this is not discussed
and either done without specifically mentioning it or sometimes the presented protocol only allows for a single unacknowledged
message, without bringing up the disadvantages of such an approach.


\paragraph{CPU bottleneck} Another bottleneck we encounter especially for small messages, is a CPU bottleneck, either at the 
sender or receiver.

A sending CPU bottleneck means we simply are not able to issue enough send requests and the RNIC is processing them faster
than we post them. This results in a predicted limit of

$$
bw \leq \frac{k}{o_{snd}}
$$


The send-receive protocol (SR) allows for multiple unacknowledged messages and avoid the previous bottleneck. 
This implementations is limited by a sending CPU bottleneck. In Figure~\ref{fig:model-bw} we can see the linearly
increasing throughput we predict until we reach a message size of 2 KB where we start to be limited by the NIC.

\paragraph{} We can avoid this bottleneck by introducing batching. The verbs API allows us to post multiple work request at 
the same time. This \emph{Doorbell batching} reduces the number of generated MMIOs~\cite{anuj-guide} and reduces the CPU load.
We measured that batching can reduce the sending CPU overhead $o_{snd}$ by up to a factor of 10. 

When introducing doorbell batching to the send-receive protocol we never seem to be bottlenecked by the sending CPU as 
can be seen for the batched measurement SR-Bat in Figure~\ref{fig:model-bw}.
Although this can drastically improve bandwidth, we will not evaluate other protocols with doorbell batching, as it is
not directly applicable to some of the protocols and for connected QPs application level batching, i.e. sending larger 
messages, is a better approach.

\paragraph{} A receiving CPU bottleneck means the receiver is unable to prepare enough receive buffers, staling the sender. 
This gives us a predicted limit of

$$
bw \leq \frac{k}{o_{rcv} + o_{post}}
$$

This results in a very similar bottleneck which is proportional to the message size. We do not encounter this bottleneck for
our 1:1 evaluation of the send-receive protocol, however when a single receiving thread handles multiple open connection 
in a N:1 communication pattern this quickly becomes the key bottleneck.

Similarly to the sending CPU bottleneck the impact can be reduced using batching.


\paragraph{Device Bottleneck} Finally if we are able to issue enough requests, and the receiver is not stalling the sender,
we are bottlenecked by either one of the involved RNICs. According to our model this results in one of these two limits

$$
bw \leq \frac{k}{o_{nsnd} + (k-1)G} \text{\quad or \quad} bw \leq \frac{k}{o_{nrcv} + (k-1)G}
$$

In both cases with increasing message size we first predict a linear increase in bandwidth which will eventually 
flatten out into the maximum possible goodput of the network link.

Looking at the batched throughput results SR-Bat in Figure~\ref{fig:model-bw}, this is qualitatively what we expect 
when being only limited by the RNIC. We are however not able to accurately predict the results for smaller messages 
using our current model. This is a limitation of our model,  possibly caused by not modeled hardware optimizations.


\pagebreak
\section{Evaluation}

We ran all our evaluations on two machines running CentOS 7 containing two Intel Xeon Gold 6152 and 384 GiB of memory.
The two nodes each contain a Mellanox ConnectX-5 (100Gbps) and are connected through a 100 Gbps switch. All measurements
have been performed using RoCE.

\subsection{Latency}

 In our latency benchmark a single client and server perform a \emph{ping-pong}. With that 
we mean that the client initiates the communication and measures the RTT. The server mirrors all received packages. 
We then take half of this RTT as our measurement of latency.

\begin{figure}[h]
\includegraphics[width=1\textwidth]{lat-msgsize.png}
\caption{Latency Evaluation}
\label{fig:plot-lat}
\end{figure}


Figure~\ref{fig:plot-lat} shows the median latency of nearly all of our protocol implementations. For all but one of these 
connections we see more or less what we have predicted given our model. They all have different base latency overheads
and then the latency grows linearly with increasing message sizes. The one connection that is standing out
and not quite adheres to our model is the buffered read connection (BR).

\paragraph{} The send-receive connection (SR) as well as the direct-read connection (DR) achieve very similar low latency,
with a base overhead of just over 2 $\mu s$. This reinforces the similarities between these two protocols and that the 
direct-write connection is essentially a reimagining of the send-receive protocol using only RDMA writes.



\paragraph{} All buffered-write protocols (BR) show a slightly increased latency compared to the send-receive protocol.
Both the write reverse (BR-Rev) as well as the not pictured write immediate implementations have a base latency overhead 
of about 2.25 $\mu s$, which is still very much comparable to the send-receive protocol. The write offset 
(BR-Off) implementation shows a higher constant overhead of about 3 $\mu s$. This is expected as this approach needs to 
issue two writes for a single message.


\paragraph{} The two read based protocols have significantly higher latency for our ping-pong evaluation. This vast
difference in latency 
is however not only caused by the read verb. For this experiment both of our read protocols have a significant 
communication overhead. Coupled with the fact that smaller reads do have a higher latency this results in the very high
base overheads we see.

The direct read protocol (DR) first sends a read request from the sender to the receive using a send operation. Only then the
receive can issue the RDMA read operation that transfers the message. So the observed base latency overhead of 
nearly 6~$\mu s$
does not only contain the read operation, but also a complete send latency for a small message.

For the buffered read protocol~(BR) the receiver needs to read from the senders memory to notice new messages. 
This means for our
latency experiment the buffered read protocol needs to issue at least two read operations per message. This results in the
very high base overhead of over 7 $\mu s$. We see another jump in latency for messages over 1 KB. This is likely cause by the
sender not having written the complete message to the buffer when the receiver issues its first polling read operation. 
This results in up to three read operations per message for large messages.

\paragraph{} We can summarize this subsection with: Simplicity is key. The most important thing to achieve low 
latency is to avoid unnecessary round-trip times and minimize the number of RDMA operations. We achieved the best
performance  using send-receive or single writes. We could however not reproduce the common assumption that RDMA writes 
are faster than send-receive.






\pagebreak

\subsection{Bandwidth}

We evaluate the single connection bandwidth of our protocol implementations for varying message sizes. Where applicable all 
protocols allow for sufficient unacknowledged messages to not run into a round-trip bottleneck. If not 
stated otherwise we report the median throughput of sending a million messages from a single sender to a single receiver
on two separate nodes.

\begin{figure}[htp]
\begin{subfigure}[b]{0.49\textwidth}
  \centering
  \includegraphics[width=1\textwidth]{write-direct-bw-msgsize.png}
  \caption{Send-Receive / Direct-Write}
  \label{fig:plot-sr-dw-bw}
\end{subfigure}
\begin{subfigure}[b]{0.49\textwidth}
  \centering
  \includegraphics[width=1\textwidth]{write-bw-msgsize.png}
  \caption{Buffer-Write}
  \label{fig:plot-bw-bw}
\end{subfigure}
\begin{subfigure}[b]{0.49\textwidth}
  \centering
  \includegraphics[width=1\textwidth]{dir-read-bw-msgsize.png}
  \caption{Direct-Read}
  \label{fig:plot-dr-bw}
\end{subfigure}
\begin{subfigure}[b]{0.57\textwidth}
  \centering
  \includegraphics[width=1\textwidth]{buf-read-bw-msgsize.png}
  \caption{Buffered-Read}
  \label{fig:plot-br-bw}
\end{subfigure}
  \caption{1:1 Bandwidth}
  \label{fig:plot-bw}
\end{figure}

Figure~\ref{fig:plot-sr-dw-bw} shows the point to point throughput for different message sizes for both the send-receive (SR)
protocol as well as the direct-write (DW) protocol. 

For the send-receive connection we see a linear increase in bandwidth with increasing message size until we hit the
maximum goodput of our network link. This indicates that we are client side CPU bound. This is confirmed by the fact that we can
improve performance for smaller messages using sender side batching which we have shown in Section~\ref{sec:model}.

The direct-write protocol however seems to be limited by the sending NIC instead or possibly by the PCIe bus, as the CPU needs
to wait for the operation to be completed and we are not limited by our ability to issue work requests like we are for the 
send-receive connection. This results in up to 30\% lower throughput for medium sized messages. We suspect that the large 
amount of returning writes interferes with outgoing writes and increases the per message overhead for the NIC. 


\paragraph{} Figure~\ref{fig:plot-bw-bw} shows the 1:1 bandwidth for all buffered write variants. We only show the measurements
using the \emph{send} acknowledger as, with one exception we will explain below, there are no significant differences
in performance between these two implementations.

Both the write immediate (BW-Imm) as well
as the write reverse (BW-Rev) implementations achieve very similar performance. They achieve slightly lower throughout
than the send-receive protocol, but show the same linear increase in performance indicating a sender CPU bottleneck. The
slightly lower throughout can be explained by the small CPU overhead of managing the ring-buffer and more importantly  
polling for tail updates.

The write offset (BW-Off) implementation shows consistently lower throughput until it also achieves
link speed for message sizes of 16 KB. This is what we would expect as this implementation needs to issue 
twice as many operations. It seems that we are bottleneck by the sending NIC overhead when the other
implementations already achieve line rate.

\paragraph{} Figure~\ref{fig:plot-dr-bw} show the bandwidth for the direct-read protocol, with using a memory fence
to issue the read and send operation at the same time, and without this fence.  The main thing we notice is that the 
fenced version achieves drastically lower throughput. The fence essentially serializes the reads and prevents the NIC 
to effectively pipeline operations. This gives us the low and linearly increasing bandwidth we observe.

While the direct read protocol actually works very differently from what we represent in our model the performance 
characteristics of the unfenced direct read connection is very similar to what we observed for previous connections. The
throughput increases linearly for small messages as we are limited by the number of request we are able to post. For larger
messages we are limited by the receiving NIC giving us this sub-linear curve until we reach the maximum goodput of our link.

For all further measurement we only focus on the unfenced implementation as the fenced version is generally unable
to achieve comparable throughput.


\paragraph{} Figure~\ref{fig:plot-br-bw} shows the 1:1 bandwidth for the buffered-read protocol. We also plot the 
\emph{mean transfer size}, so the data transfered by a single read. This can vary greatly for this protocol as it periodically
fetches the complete sending ring-buffer. For this reason we also plot the mean bandwidth instead of the median bandwidth 
as the data is not normally distributed.

As expected with increasing message size the mean transfer size also increases linearly and with that the 
total bandwidth also increases until we hit a maximum at around 70 Gbit/s and a transfer size of about 100KB. After that 
the performance starts to degrade. \comment{Why? I can't explain it really. I'm having a hard time in general to explain 
this graph}

We can also see that we achieve significantly better performance for very small messages compared to the other protocols we
look at in this thesis. This is a result of the automatic batching behaviour of this protocol. 
Performance of the buffered-read
protocol in a single connection situation could be drastically improved by limiting the maximum transfer size and 
preemptively issuing reads to fill up the pipeline and not having to explicitly wait for each
read to complete.


\begin{figure}[h]
\includegraphics[width=1\textwidth]{write-bw-rev-anom.png}
\caption{Buffered write reverse bandwidth for message sizes around 4 KB with read acknowledgements}
\label{fig:plot-write-rev-anom}
\end{figure}

\paragraph{Write Reverse Anomaly} During our evaluation we encountered a anomaly we cannot fully explain at
this point. As seen in Figure~\ref{fig:plot-write-rev-anom}, when using a \emph{write reverse} sender and a 
\emph{read} acknowledger we see a significant drop in performance when sending messages that are slightly larger
than 4090 bytes. It is worth noting that when adding the 6 byte overhead from the protocol explained in 
Section~\ref{sec:conn:write:sender} this happens to be exactly when the write is larger than 4 KB, 
which is both the pagesize and MTU. Bandwidth then seems to linearly increase and will drop again very similarly 
for all multiples of 4 KB.

Interestingly this cannot be observed with other sender implementations and the effect is greatly
reduce when using send acknowledgements instead of read based acknowledgements.

We suspect this anomaly is caused by our unusual write direction which causes suboptimal NIC cache usage and 
results in many TLB cache misses while writing. But this is just one possible explanation and further research is 
necessary to fully explain these results.

For all other plots we purposely avoid to exactly hit this window for the \emph{write reverse}
sender but will send slightly smaller messages.


\paragraph{} To maximize 1:1 bandwidth, reducing the number of necessary RDMA operations seems to be important.
The best performing protocols are the send-receive 
protocol as well as the buffer-write implementations that only issue a single write. Compared to our latency evaluation
however the difference are not as severe. 
Protocols like direct-read or direct-write, which can achieve true zero-copy capabilities, can still achieve decent 
throughout and if we are handling messages larger than 16 KB almost all approaches can achieve line rate.








\pagebreak
\subsection{N:N}


We can see that when we have single point to point connection, we are almost always limited by the sender and the 
amount of requests we are able to issue. In practice however we usually do not have a such a simple setup, but we
need to send to and receive from multiple different nodes. This means each node needs to handle multiple open
connections. 

To evaluate the performance of our protocols with multiple open connections we again only use two nodes, however  
on each node we run $T$ threads. Each thread $t_k$ opens a connection with the corresponding 
thread on the other node, giving us a total of $T$ connection sharing the same NIC. We evaluate the throughput for 
three different message sizes which had different performance characteristics in our single threaded evaluation: 16 bytes, 
which was usually heavily bound by how fast we could post send requests, 512 bytes, which was also normally limited by the 
sender but less extremely, and 8192 bytes, which is limited by the actual device bandwidth for almost all protocols. 

We do not perform any sender side batching but allow for sufficient unacknowledged messages to keeping the pipeline full.
In our plots we report the sum of all connection throughputs.

\begin{figure}[ht]
  \centering
\begin{subfigure}[b]{0.49\textwidth}
  \centering
  \includegraphics[width=1\textwidth]{send-bw-threads.png}
  \caption{Send-Receive}
  \label{fig:plot-sndrcv-bw-thread-nosrq}
\end{subfigure}
\begin{subfigure}[b]{0.49\textwidth}
  \centering
  \includegraphics[width=1\textwidth]{send-bw-srq-threads.png}
  \caption{Send-Receive with SRQ}
  \label{fig:plot-sndrcv-bw-thread-srq}
\end{subfigure}
  \begin{subfigure}[b]{0.48\textwidth}
  \centering
  \includegraphics[width=1\textwidth]{write-direct-bw-threads.png}
  \caption{Direct-Write}
  \label{fig:plot-wdir-bw-threads}
  \end{subfigure}
\caption{N:N Send-Receive / Direct-Write Bandwidth}
  \label{fig:plot-sndrcv-bw-thread}
\end{figure}


\paragraph{} Figure \ref{fig:plot-sndrcv-bw-thread-nosrq} shows the N:N throughput for the send-receive protocol.
We can see that for large messages we keep being bottlenecked by the network bandwidth of 100 Gbit/s. For smaller
messages the throughout first increases linearly until we hit a bottleneck. Interestingly for a message size 
of 16 bytes we are able to send over twice the amount of messages per second compared to a message size of 512 bytes.
We suspect this to be caused by NIC level optimizations for small messages such as inline receiving~\cite{anuj-guide} which 
is supported by Mellanox NICs up to 64 bytes. This bottleneck we encounter seems to be the total throughput of the 
receiving NIC. This is significantly higher than our maximum throughput seen for a single connection, this can be 
attributed to the usage of multiple processing units~\cite{anuj-guide}


Figure \ref{fig:plot-sndrcv-bw-thread-srq} shows the same data when using a shared receive queue (SRQ) to share memory
between the connections. In this case we seem to be limited by the rate at which the receiver is able to
repost receive buffers. We
had to limit all senders to about 1.8 MOp/s to not run into RNR erros.

We later show that when using SRQs we are limited a maximum of 2 MOp/s. So even with more optimized receive buffer management
this is drastically lower than without any resource sharing and heavily limits the performance for small messages. This seems
to be a limit of the receivers NIC.

\paragraph{} Figure~\ref{fig:plot-wdir-bw-threads} shows the N:N bandwidth for the direct-write protocol. When using 8192 
byte messages, we are again limited by the maximum link speed. When using 512 byte messages we seem to bottleneck at around 
70 Gbit/s, and for small messages we are capped at around 6 Gbit/s, which is very close to the performance of the send-receive
protocol.

However the 70 Gbit/s limit for 512 byte messages is significantly less than the 80 Gbit/s maximum we saw for the
send-receive protocol. We also see a slightly lower goodput for 8192 byte messages. We assume this is caused by the 
large amount of returning write, limiting the NICs maximum throughout. We can see very similar performance for the
write offset implementation in Figure~\ref{fig:plot-write-bw-thread-512}. This reinforces our assumption that the returning write
for each message is impacting our throughput, as the write offset protocol also needs to issue two writes per message.




\begin{figure}[ht]
  \centering
\begin{subfigure}[b]{0.49\textwidth}
  \centering
  \includegraphics[width=1\textwidth]{write-bw-threads-16.png}
  \caption{Message size 16 bytes}
  \label{fig:plot-write-bw-thread-16}
\end{subfigure}
\begin{subfigure}[b]{0.49\textwidth}
  \centering
  \includegraphics[width=1\textwidth]{write-bw-threads-512.png}
  \caption{Message size 512 bytes}
  \label{fig:plot-write-bw-thread-512}
\end{subfigure}
\begin{subfigure}[b]{0.49\textwidth}
  \centering
  \includegraphics[width=1\textwidth]{write-bw-threads-8192.png}
  \caption{Message size 8192 bytes}
  \label{fig:plot-write-bw-thread-8192}
\end{subfigure}
  \caption{N:N Buffered-Write Bandwidth}
  \label{fig:plot-write-bw-thread}
\end{figure}


\paragraph{} Figure \ref{fig:plot-write-bw-thread} shows the total throughput of all buffered write implementations
with varying number of connections and the three different message sizes.

We can again see that for large messages we keep being bottlenecked by the network bandwidth of 100 Gbit/s, regardless of 
the sender, however the write offset (BW-Off) implementation shows a small overhead and does not quite achieve the same 
throughput as the other two implementation. 

For smaller messages we see a linear increase in throughput with increasing number of connections until we hit a bottleneck. 
There does seem to be a fair amount of noise,
but for both message sizes the write immediate (BW-Imm) and write reverse (BW-Rev) implementation achieve similar performance,
while the write offset implementation is significantly slower. This can again be explained by the additionally issued write, 
which in this case  gives us a more pronounced overhead as we are bottlenecked by the number of operations the NIC can process.


Compared to the previously evaluated send-receive protocol, both the write immediate as well as write reverse implementation
achieve very 
similar bandwidth, while the results of the write offset implementation are very comparable to the direct-write protocol.


\begin{figure}[ht]
  \begin{subfigure}[b]{0.49\textwidth}
  \centering
  \includegraphics[width=1\textwidth]{dir-read-bw-threads.png}
  \caption{Direct-Read}
  \label{fig:plot-dirread-bw-threads}
  \end{subfigure}
  \begin{subfigure}[b]{0.49\textwidth}
  \centering
  \includegraphics[width=1\textwidth]{buf-read-bw-threads.png}
  \caption{Buffered-Read}
  \label{fig:plot-bufread-bw-threads}
  \end{subfigure}
  \caption{N:N Bandwidth Read-based Protocols}
\end{figure}

\paragraph{}Figure~\ref{fig:plot-dirread-bw-threads} shows the direct-read protocols N:N performance. The results look very similar to 
other connections. There is a linear increase in performance, both by being able to post more work request for smaller 
messages and being able to utilize more NIC processing units.~\cite{anuj-guide}

When using 8192 byte messages, we are limited by the maximum link speed. When using 512 byte messages we hit a bottleneck at
around 70 Gbit/s, which is in line with what we observe with other protocols that need to issue two operations per message, 
like the direct-write or write-offset connection.



\paragraph{} Figure~\ref{fig:plot-bufread-bw-threads} shows the buffered-read protocols N:N performance. We see drastically 
improved performance compared to the single connection evaluation. This is mainly caused by the fact that we now have multiple
concurrent active read operations. Through the inbuilt message batching we achieve  significantly higher throughput for small
messages compared to the other connection types we evaluated. Throughput for all message sizes grow linearly with increasing 
number of connections and are only limited by the line rate or eventually by the individual copying of buffers at the sender 
and its function call overhead.

This is a drastically different performance profile compared to our other protocols. 
The aggressive sender side batching allows
for very high bandwidth, but at the cost of increased latency.


\paragraph{} The main takeaway when designing protocols for N:N communication patterns is to reduce the number of operations
per message. We saw very similar throughput for all connections with a single operation per message and two operations per
message respectively. Interestingly it does not seem to matter in which direction these operations are performed. The 
direct-write protocol with its returning writes performed very similarly to the write-offset or direct-read protocol.

One thing the buffered read connection has shown us is that to achieve optimal bandwidth for small messages sending side and
whenever possible application level batching is necessary. And we also saw that SRQs, while saving memory usage, have 
significant performance limitations.



\pagebreak
\subsection{N:1}
One of the most prevalent communication pattern is the N:1 configuration, where a single server handles the messages
of multiple clients. 

For our evaluation we only use two nodes. On the sending node we have $N$ threads that send messages to a
single receiving thread on the other node. If not stated otherwise this receiving thread will simply 
round robin over the $N$ open connections. 

As we did for the N:N experiments, we evaluate the throughput for three different message sizes: 16 bytes, 
512 bytes, and 8192 bytes. We do not perform any sender side batching but allow for sufficient unacknowledged messages 
and in our plots we report the sum of all connection throughputs.


\begin{figure}[ht]
  \centering
\begin{subfigure}[b]{0.49\textwidth}
  \centering
  \includegraphics[width=1\textwidth]{send-bw-n1.png}
  \caption{Send-Receive}
  \label{fig:plot-sndrcv-bw-n1-nosrq}
\end{subfigure}
\begin{subfigure}[b]{0.49\textwidth}
  \centering
  \includegraphics[width=1\textwidth]{send-bw-srq-n1.png}
  \caption{Send-Receive with SRQ}
  \label{fig:plot-sndrcv-bw-n1-srq}
\end{subfigure}
  \begin{subfigure}[b]{0.48\textwidth}
  \centering
  \includegraphics[width=1\textwidth]{write-direct-bw-n1.png}
  \caption{Direct-Write}
  \label{fig:plot-wdir-bw-n1}
  \end{subfigure}
\caption{N:1 Send-Receive / Direct-Write Bandwidth}
  \label{fig:plot-sndrcv-bw-n1}
\end{figure}



\paragraph{} Figure~\ref{fig:plot-sndrcv-bw-n1-nosrq} shows the throughput for the send-receive protocol and a single receiver. We
use the single receive approach we described in Section~\ref{sec:conn:send}, which allows us to route all completion events for 
multiple QPs to a single receiving thread. To prevent Reader-Not-Ready (RNR) errors, which happen when the receiving CPU is 
unable to repost receive buffers quickly enough, we limit the sender to a stable sending rate.

For 16 byte messages we seem to be limited at around 11 MOp/s, while for 512 bytes we limited at around 16 MOp/s. Both bottlenecks
are caused by the receiving CPU. We expect the difference in sustainable message rates to be the result of inline receives for 
16 byte messages, which actually has detrimental effects on throughput, as this causes additional overhead for the CPU which 
is already the bottleneck in this situation.

However for both smaller message sizes we see a drop in performance when further increasing the number of sender. We explain 
this by increased cache misses, caused by having to access more QPs and the linearly growing number of receive buffers.

\paragraph{} Figure~\ref{fig:plot-sndrcv-bw-n1-srq} shows the same plot while using a shared receive queue for all QPs. For
large messages we are again only limited by the link speed.

For small messages of 32 bytes or lower we see similar throughout as without using a shared receive queue. We are still
limited by the receiving CPU. We do however not see any performance drops when using an increasing amount of senders. This
would fit our explanation of cache misses for receive buffers, as when we are using a SRQ the number of receive buffer stays
constant.

More interesting are message sizes above 32 bytes that are not limited by the link speed. Without the receive optimizations 
for very small messages we seem to be limited at 2 MOp/s
when using an SRQ. Interestingly this effect does not show up when only
one QP is using the SRQ. We expect this to be some kind of locking or atomic operation of the NIC itself. This results in a
major bottleneck for any implementation using shared receive queues.

\paragraph{} Figure~\ref{fig:plot-wdir-bw-n1} shows the direct-write protocols throughput. Similarly to the 
send-receive protocol we are very much limited by the receiving CPU.

Very large messages achieve the maximum goodput,  while smaller messages seem to be limited by the reposting speed of the 
receiver at around 4 MOp/s, which is already a bottleneck with two open connections.

This bottleneck gives us performance caps for 512 byte messages at around 15 Gbit/s and at 0.5 Gbit/s for 16 byte messages.
There seems to be a very slight reduction in throughput when further increasing the number of open connections, which we 
attribute to increased cache misses.

The maximum throughput could be be improved by batching returning writes, but as it stands the direct-write connections
achieves significantly worse performance than the send-receive protocol.



\begin{figure}[ht]
  \centering
\begin{subfigure}[b]{0.49\textwidth}
  \centering
  \includegraphics[width=1\textwidth]{write-bw-n1-16.png}
  \caption{Message size 16 bytes}
  \label{fig:plot-write-bw-n1-16}
\end{subfigure}
\begin{subfigure}[b]{0.49\textwidth}
  \centering
  \includegraphics[width=1\textwidth]{write-bw-n1-512.png}
  \caption{Message size 512 bytes}
  \label{fig:plot-write-bw-n1-512}
\end{subfigure}
\begin{subfigure}[b]{0.49\textwidth}
  \centering
  \includegraphics[width=1\textwidth]{write-bw-n1-8192.png}
  \caption{Message size 8192 bytes}
  \label{fig:plot-write-bw-n1-8192}
\end{subfigure}
  \caption{Bandwidth with varying number of threads and a single receiver}
  \label{fig:plot-write-bw-n1}
\end{figure}

\paragraph{} Figure~\ref{fig:plot-write-bw-n1} shows the N:1 throughput for all buffered-write protocols as 
well as the two shared-write implementations.

For large messages of size 8196 throughput is again limited by the link speed for all buffered-read implementations, 
with the same slight overhead for the write offset (BW-Off) implementation we have seen in all previous bandwidth plots.
For smaller messages the buffered-write protocols are limited by the receivers CPU. We can avoid any 
RNR errors for the write immediate sender by limiting the ring-buffer size. This way we do not have to artificially limit 
the sender as we did for the send-receive protocol.

For both 16 byte as well as 512 byte messages, the write reverse \mbox{(BW-Rev)} implementation achieves up to 20\% higher throughout
compared to the write immediate (BW-Imm) protocol. This can be explained by the additional receive buffer reposting the 
receiving CPU needs to perform when using write with immediate. 

The write offset (BW-Off) implementation achieves similar performance to BW-Imm. Interestingly BW-Off seems to reach
slightly higher performance when opening more than 6 concurrent connections with both 512 byte as well as 16 byte
sized messages. As of
writing this, we do not have a clear explanation for it. Our current best guess is, that with more than 6 connections the 
metadata is spread over more than one cache line. This can result in less cache invalidation and in turn in lower receiver 
CPU usage and better overall performance. Further research into this is necessary. \comment{Veery rough guess}


\paragraph{} Figure~\ref{fig:plot-write-bw-n1} also evaluates the shared write protocol (SW). We can see that we are strongly limited
by our two phase approach for all message sizes. That means we see a linear increase in bandwidth with increasing number of 
senders as we are able to issue more requests in parallel.

For the large message size of 8 KB the total throughput increases linearly until we hit the line rate with 10 
active senders. The use of device memory \mbox{(SW-DM)} allows us to saturate the link with only 9 connections.
This is not a 
large difference and we are able to take full advantage of the bandwidth with or without the usage of device memory.

Smaller messages give use more interesting results. For a message size of 512 bytes we first see a linear 
increase in bandwidth for both versions with and without usage of device memory. The version without device memory however
caps at around 7.5 Gbps, which is exactly what we predicted given our \emph{fetch and add} micro-benchmark that gave us a
peek throughput of 2 MOp/s. The version utilizing device memory for the metadata achieves higher throughput with more
senders and we expect it to theoretically reach a maximum throughput of around 30 Gbps with enough concurrent senders.

We see very similar results for the 16 bytes messages. The version performing fetch and add on RAM peaks at about 0.25 Gbps, 
while the version operating on device memory is able to achieve higher throughput.



\begin{figure}[ht]
  \begin{subfigure}[b]{0.49\textwidth}
  \centering
  \includegraphics[width=1\textwidth]{dir-read-bw-n1.png}
  \caption{Direct-Read}
  \label{fig:plot-dirread-bw-n1}
  \end{subfigure}
  \begin{subfigure}[b]{0.49\textwidth}
  \centering
  \includegraphics[width=1\textwidth]{buf-read-bw-n1.png}
  \caption{Buffered-Read}
  \label{fig:plot-bufread-bw-n1}
  \end{subfigure}
  \caption{N:1 Bandwidth Read-based Protocols}
\end{figure}

Figure~\ref{fig:plot-dirread-bw-n1} shows the direct-read protocols throughput with a single receiver and varying number of senders.
We are again very much limited by the receiving CPU in this case, as the receiver already needs to do the heavy lifting for 
this protocol. 

For smaller messages there is no throughput improvements at all with increasing number of connections as we already have 
been limited by the ability of the receiving CPU to post work requests. For very large messages we see small improvements
in throughput. This is caused by the ability to use multiple NIC processing units.

This communication pattern does not seem to be good fit for this protocol as it is always limited at around 
2.6 MOp/s by the receiving CPU speed. This gives us the 10 Gbit/s and 0.3 Gbit/s bottlenecks for the 512 byte and
16 byte size messages respectively.

\paragraph{} Figure~\ref{fig:plot-bufread-bw-n1} shows the buffered-read protocols N:1 throughput. It achieves basically the 
same performance as when using a single connection. This is exactly what we expect as receiving from multiple connections
behaves the same way as receiving from a single connection.

There seems to be performance degradation when increasing the number of senders. This can for one be explained
by increased cache misses and more importantly for large messages the same performance drop we see in the single connection 
benchmark when further increasing the transfer size.


\paragraph{} Unsurprisingly when developing a protocol for a N:1 communication it is vitally important to reduce the 
involvement of the receiving CPU in the transmission. This makes read based protocols unsuitable. Interestingly in our
evaluation the send-receive protocol outperformed both the direct as well as buffered-read protocols. But we need to 
keep in mind that we should be able to drastically improve the direct-write protocol's performance by reposting
buffers in batches or by redesigning this reposting entirely to reduce the number of returning writes.

Also for the send-receive protocol to be stable enough for production use we would need to add some kind of acknowledging to
avoid RNR errors, which should further level the playing field and result in similar performance for send and write based 
protocols.

Finally we again saw that resource sharing does not come for free. Both the shared write as well as shared receive queue based
protocols are significantly slower than their unshared counterparts. 










\pagebreak
\section{Conclusion}



\pagebreak

\bibliographystyle{unsrt}
\bibliography{references}{}

\end{document}
