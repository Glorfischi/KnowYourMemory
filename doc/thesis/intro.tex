\section{Introduction}

Remote Direct Memory Access (RMDA) is a powerful communication mechanism that offers the potential for exceptional performance.
RDMA allows one machine to directly access the memory of a remote machine across the network without the interaction of the 
remote CPU. This gives developers a plethora of options to implement communication protocols. However using these options 
effectively is non trivial and the observed performance can vary greatly for seemingly minor differences. 

Existing research either primarily focus on evaluating very low level verb performance~\cite{anuj-guide} or focus strongly on 
Remote Procedure Calls (RPCs)~\cite{eval-mpp} often comparing the observed performance to using remote data 
structures~\cite{fasst, rpc-vs-rdma}. Nearly all of them employ very simply message passing protocols using either 
send receive or writes with circular buffers~\cite{rdma-fast-dbms} or \emph{mailboxes}~\cite{ziegler2020rdma} and do not take full advantage of 
features offered by modern RDMA-capable network controllers. Further hardly any work looks into the usage of shared receive
queues, memory fences, or atomics for resource sharing.

\paragraph{}In this thesis we implement and evaluate various different message passing protocols. We show that there are a 
lot of ways to implement data exchange connections using less used RDMA features such as \emph{shared receive queues}, \emph{reads}, 
\emph{memory fences}, and \emph{RDMA atomics}. We also show that even more common connections such as ringbuffer based protocols 
can be implement in multiple ways giving us different performance characteristics and features.

\begin{itemize}
  \item We focus on implementing message passing protocols, without limiting us to RPCs. We believe this gives engineers 
    building blocks to develop more sophisticated protocols without micro-benchmarking basic verbs.
  \item We define other connection features outside of raw performance which have been relevant for applications such as 
    efficient resource usage or fairness.
  \item We implement and evaluate different message passing protocols. We reason why we explicitly implemented these protocols
    and evaluate them in both single threaded and multi-threaded settings.
  \item We provide a performance model to better understand the observed performance.
\end{itemize}

\paragraph{}We hope that this work, with 
implementations of more unusual communication protocols, inspires system designers to develop 
new protocols that make more extensive use of modern RNICs.


