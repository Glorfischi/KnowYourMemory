\section{Shared Write}
\subsection{Protocol}
\subsection{Device Memory}
\subsection{Evaluation}
\todo{Info on system etc.}

\subsubsection{Latency}

\begin{figure}[h]
\includegraphics[width=1\textwidth]{write-atomic-lat-msgsize.png}
\caption{Evaluation of the Shared Write latency between two nodes}
\label{fig:plot-write-atomic-lat}
\end{figure}

Figure \ref{fig:plot-write-atomic-lat} shows the latency of our \emph{Shared Write} protocol between two nodes. 
We again show half the ping-pong round-trip time. We can clearly see that we have a large constant overhead, caused
by the reserving phase of over $2L$ independent of the message size.

We are reduce this constant overhead by about 0.25 $\mu$s by using device memory for the connections metadata. This is 
in line with what we expect given our micro benchmark, and is caused by the receiving NIC not having to access the 
receivers RAM.


\subsubsection{Bandwidth}

\begin{figure}[h]
\includegraphics[width=1\textwidth]{write-atomic-bw-msgsize.png}
\caption{Evaluation of the Shared Write bandwidth between two nodes}
\label{fig:plot-write-atomic-bw}
\end{figure}

In figure \ref{fig:plot-write-atomic-bw} we can see the throughput of the \emph{Shared Write} protocol for a single 
connections, with varying message size. 

\todo{This is very slow, we would need to redesign the connection a little, as right now we can only support 2 unack 
messages}

show linear


\subsubsection{Resource Sharing}

\begin{figure}[h]
\includegraphics[width=1\textwidth]{write-atomic-bw-threads.png}
\caption{Shared Write bandwidth with a single receiver and varying number of sender}
\label{fig:plot-write-bw-unack}
\end{figure}





