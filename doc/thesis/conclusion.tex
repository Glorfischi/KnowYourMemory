\section{Conclusion}

This thesis described and implemented various RDMA based protocols for data exchange, we studied their advantages and 
drawbacks and evaluated them for multiple communication patterns.

In contrast to most related work we did not focus on low level verb performance or on remote procedure calls, but instead 
focused on different approaches to design a message passing protocol. These protocols can act as building blocks for 
system designers to develop new and more complex protocols.

\paragraph{} We implemented and evaluated widely used protocols such as a send-receive based protocol and multiple different
ring-buffer based write protocols. We extended them to allow resource sharing between multiple connections, with shared 
receive queues and atomic operations respectively. We then introduced other rarely used approaches that use RDMA reads or 
multiple writes with the goal of creating more flexible protocols and reducing the need for additional copying.

We looked at rarely used RNIC features such as shared receive queues, on device memory, and RDMA memory fences and we looked
at their capabilities and drawbacks.


\paragraph{} We analysed the presented protocols and evaluated more then just raw performance, we looked at other
features that can be critical for applications.  These features include achieving effective memory usage by allowing 
for variable message sizes, avoiding additional copying by being truly zero-copy, or one slow processing message not
being able to stall the whole connection.

\paragraph{} We introduced a new on LogGP based performance model for RDMA based message passing protocols, that allows us to 
better understand the evaluated protocols. We discussed the often ignored limitation of protocols that do not allow for enough
concurrent requests, severely limiting the capabilities of an RDMA-enable network. We also show that for high speed networks
even with the heavy NIC offloading of RDMA, the systems CPU is often the bottleneck as it is unable to issue enough work requests to
keep the RNIC busy if the protocol is not carefully designed.


\paragraph{} This work shows that while there has been much work in designing and optimizing certain RDMA data
exchange protocols, there is still room for many interesting protocols that take full advantage of modern RDMA-enabled NICs.
We think that systems engineers can significantly improve overall system performance by putting more thought and research in
RDMA protocol design and how it interacts with the rest of the system, instead of relying on the same widely used protocols.
