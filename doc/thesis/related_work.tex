\section{Related Work}

In recent years there has been a lot of work on making effective use of RDMA and on avoiding pitfalls of designing  
RDMA enabled protocols. Most of it focuses on either low level verb performance or design and evaluate much higher level
abstractions that mostly focus on RPC performance or the performance of a whole system.

\paragraph{} We give an overview of related work that discuss how  RDMA can be used effectively as well as an overview of 
some recent systems that try to utilize these capabilities.

\subsection{Using RDMA Effectively}

Design Guidelines for High Performance RDMA Systems~\cite{anuj-guide} provides a very low level guide on how to
effectively use RDMA enabled networks. It focuses on specific optimizations to make and pitfalls to avoid in order to 
achieve the best possible performance. 

\paragraph{} There are many papers focusing on RPCs. Some try to systematically compare RPC protocol
approaches~\cite{ziegler2020rdma,Huang2019AnEO}.  These gives us valuable insight in verb performance and scalability, but 
they all use very simplified protocols and a very limited design space.

The Remote Fetching Paradigm~(RFP)~\cite{rfp} exposes an RPC interface while utilizing RDMAs asymmetric performance 
characteristics. In RFP, clients write their request to a request buffer on the server. The server processes the request
and writes the result to a local buffer. The client then polls this sever-local buffer for the response. This approach reduces 
server side load and makes it \emph{passive} from a network interaction standpoint. A feature we also look at in this thesis.

ScaleRPC~\cite{scal-rdma-rpc} and other works~\cite{fasst, rfp, herd} address the scalability of RDMA for RPCs when using 
\emph{Reliable Connections}. They point out that the main scalability problem of reliable connections stem from limited 
RNIC and CPU caches, resulting in cache thrashing with a large amount of active Queue Pairs(QPs). ScaleRPC uses RDMA writes to 
transmit both requests as well as responses. It addresses scaling problems with temporal slicing.



\paragraph{} There has been quite a lot of research into designing fast and scalable RDMA system. This thesis aims to take
a step back and take a broader look at RDMA based protocols for data exchange. A recent master thesis of the University of 
Waterloo~\cite{sharma2020design} saw a similar need for a design space analysis. They look at the design space of \emph{flow structures},
a much narrower protocol definition compared to our definition. But the work provides a similar analysis of possible
protocols and their features.


\subsection{RDMA Systems}

There have been a lot of work on designing and evaluating new systems that take full advantage of RDMA features.

Many papers focus on optimizing RDMA enabled key-value stores. There are designs like Nessie~\cite{nessie} that is 
client-driven and only relies on one-sided verbs like RDMA read, write, and atomics to issue \code{GET}, \code{PUT}
or \code{DELETE} operations without server side polling.
Systems like Pilaf~\cite{pilaf} and FaRM~\cite{farm} also use RDMA reads to perform \code{GET} operations. But both use 
RPC like approaches for \code{DELETE} and \code{PUT}. While Pilaf relies on send and receive verbs, FaRM uses a ring buffer
based protocol using RDMA writes.
HERD~\cite{herd} uses a fully server-driven approach and issues all key-value operations through an RPC interface. For this
RPC protocol clients issue requests using RDMA writes to server-polled memory region. The server processes the request and
replies using the RDMA send verb and unreliable datagrams.

Other works design new database systems~\cite{dbrackjoin} or propose new approaches to message brokers~\cite{broker}
that take full advantages of RDMA features. 

\paragraph{} There are many updates to existing systems that replace existing sockets with RDMA 
connections~\cite{memcached, hbase, hdfs}. This can result in a significant boost in performance. RDMA-Memcached~\cite{memcached} 
uses this approach and achieves a performance improvement of over four times compared to using traditional TCP/IP sockets.

There are however also newly designed systems that rely on classical message passing interfaces. For example 
PolarFS~\cite{polarfs} uses a hybrid approach of using Send/Receive as well as Read/Write, which is reminiscent of 
our \emph{direct read} protocol we introduce in Section~\ref{sec:conn:direct_read}. This is also a good example of a system 
that does not have RDMA at its core, but rather
uses it as another features to speed up the system. 

This thesis is a good starting point for engineers building similar systems to design an appropriate RDMA-based protocol.



