
\section{Direct Read} \label{sec:conn:direct_read}
\subsection{Design}
\todo{Description of direct read}

\paragraph{Latency Analysis}

We estimate the latency $t_{dr}$ of transfering a single message $m$ of size $k$ from the sender to the receiver. 
A detailed sequence diagram of what exactly is happening is in figure \ref{fig:seq-direct-read}. 

\begin{enumerate}
  \item The transfer is initiated by the sender sending a \emph{read request} using an inline send. This means the sender
    will post an inline send using MMIO, which will be transmitted to the receiver, which will issue two DMAs and will have
    to poll on the CQ. The complete transfer of this read request gives us a constant overhead of $o_{rr}$. By looking at the
    components of this transfer we know that $o_{rr} \geq o_{sil} + L + o_{rcv} + o_{cq}$
  \item The receiver will then issue a read. This includes posting a WR, transmitting a request to the sender, which will start
    a DMA to read the payload. This is another constant overhead of $o_{rd} \geq o_{snd} + L$.
  \item The payload is then transmitted to the receiver in essentially the same way as in the other protocols.
  \item As soon as the receiver will write the incoming payload to the specified buffer, generate a 
    \emph{Work Completion Event (WQE)}, and the receiving CPU will have to poll the \emph{Completion Queue (CQ)}. 
    This will again give us a constant overhead of $o_{rcv} + o_{cq}$. \todo{This is not quite true $o_{rcv}$ contains the 
    consumption of a receive buffer}
\end{enumerate}



\begin{figure}[!ht]
\begin{center}
\begin{tikzpicture}[node distance=2cm,auto,>=stealth']
  \seqnode{B_cpu}{RAM};
  \seqnode[left of=B_cpu]{B_nic}{NIC};
  \hseqnode[right of=B_cpu, node distance=1.5cm]{B_acpu}{};
  \seqnode[left of=B_cpu, node distance=7cm]{A_cpu}{CPU / RAM};
  \seqnode[right of=A_cpu]{A_nic}{NIC};
  %
  \msg{A_cpu}{A_nic}{.25}{inline WR MMIO}
  \msg{A_nic}{B_nic}{.3}{transfer read request}
  \msg{B_nic}{B_cpu}{.35}{request DMA}
  \msg[below]{B_nic}{B_cpu}{.4}{WQE DMA}

  \fetch{B_acpu}{B_cpu}{.45}{poll CQ}
  \msg{B_acpu}{B_nic}{.6}{read WR MMIO}

  \msg{B_nic}{A_nic}{.65}{transfer read request}
  \msg{A_cpu}{A_nic}{.72}{payload DMA}
  \msg{A_nic}{B_nic}{.8}{transfer payload}
  \msg{B_nic}{B_cpu}{.85}{payload DMA}
  \msg[below]{B_nic}{B_cpu}{.9}{WQE DMA}
  \fetch{B_acpu}{B_cpu}{.95}{poll CQ}

  \end{tikzpicture}
\end{center}
\caption{Direct Read sequence}
\label{fig:seq-direct-read}
\end{figure}


\begin{align*}
  t_{dr} &\geq o_{rr} + o_{rd} + (k-1)G + L  + o_{rcv} + o_{cq}\\
         &\geq o_{sil} + o_{snd} + 3L  + (k-1)G  + 2o_{rcv} + 2o_{cq}
\end{align*}


\subsection{Evaluation}
