
\section{Direct Read} \label{sec:conn:direct_read}

In Section~\ref{sec:conn:direct_write} we discussed how we can possibly avoid an additional copy at the receiver by giving 
the sender information which allows him to potentially write the data to the correct final memory location. The next logical
step is to let the receiver decide for each message where to write it to. We can achieve this by our implementation of a
\emph{Direct Read Connection}.

\paragraph{} The core idea of a direct read protocol is that instead of directly sending a message through a send or write 
verb, the sender simply notifies the receiver of a new message and where it is located, we will call this a \emph{Read Request}.
The receiver then issues an RDMA read operation to directly read this message to the correct location.

By allowing the sender to attach additional domain specific information to this read request, the receiver can directly 
move this data to the correct space in a data structure, or potentially even directly to NVMe storage for certain applications.
\comment{I remember correctly this is possible right? I had a hard time finding a relevant paper}


\subsection{Protocol}
The sender interface is again unchanged and consists of a \code{SendAsync()} method which takes a buffer containing the
message and a \code{Wait()} method that waits until the transfer is completed.

\paragraph{} Instead of sending the complete buffer, the sender only sends a small \emph{Read Request} using an inline send,
containing the location of the buffer. It is then the job of the receiver to access this buffer.

\paragraph{} To wait for the transfer to be completed, and for the buffer to be able to be reused, we can not simply wait 
for the completion event of the send, like we do for the send or write based connections. We need to wait for the receiver 
to explicitly signal that the buffer was transfered. We append a signaling byte at the end of the send buffer. 
When sending this byte will be set to 0 and we can wait for the transport to be completed by polling this byte until the 
receiver will update it.

This push based implementation introduces little additional complexity, but there are other ways to implement such 
signaling. The signaling bit forces us to use a specific memory arrangement, which could prevent us to send data directly 
from certain data structures. In such cases a pull based approach or an implementation using send might be a better approach
and depending on the implementation could actually result in better performance.

\paragraph{} The receiver polls the receive queue and receives the \emph{Read Request}. It then directly reposts the associated 
receive buffer. Using the information in the Read Request it issues an RDMA read to transfer the message. 


It is crucial that we do not block until the read is completed to get reasonable performance.  This means the receiver has a
slightly different interface that the previously presented connections. We split the receive call into a 
\code{RequestAsync} and a \code{Wait} method. The \code{RequestAsync} takes a receive buffer
to read into. It will wait for an incoming read request and issue the corresponding read. It uses the same increasing 
\code{wr\_id} approach we use for sending with which the \code{Wait} method can wait for the read to complete. This approach
allows us to pipeline receives the same way we pipeline sends.

\paragraph{} As soon as the transfer is complete, the receiver updates the corresponding signaling bit using an RDMA write.

\subsection{Fences}

To signal the end of the transmission, the receiver needs to issue a write as soon as the read is completed. One way to 
implement this is by issuing the write as part of the receivers \code{Wait}. This is the approach we implemented for our
base version.

One idea we had to optimize this is by issuing the write at the same time as we issue the read. This however introduces a 
problem. While RDMA guarantees that two consecutively issued writes and sends will happen in the issued order, 
it will not guarantee us that reads issued before writes will be completed before we can observe the write~\cite{rdma-reference}.
This results in a data race when we issue a read directly before a write. To still be able to post these two operations 
at the same time, we need to add an \code{IBV\_SEND\_FENCE} flag to the write request. When we add a fence indicator to a write
request its processing will not begin until all previous read and atomic operations on the same QP  have been completed. This 
allows us to issue both the read as well as the acknowledging write at the same time.

\subsection{Features}

The Direct Read connection is a protocol with the main goal of achieving  \emph{True Zero Copy} capabilities as 
the receiver itself can read the message to its final place, preventing any additional 
copying. 

The protocol is also \emph{Non-Blocking} and achieves efficient memory usage though \emph{Variable Message Sizes} and not 
having any fixed buffers. With that \emph{Resource Sharing} follows as multiple connections can reuse the same receive and 
send buffers. It also allows us to use \emph{Interrupts} from completion events
giving us low CPU utilization in low load situations as no component needs to constantly poll.

All these features make this protocol viable, especially when handling large messages.
